\documentclass[12pt,preprint]{aastex}
\usepackage{cite,natbib}
\usepackage{epsfig}
\usepackage{cases}
\usepackage[section]{placeins}
\usepackage{graphicx, subfigure}

\begin{document}

We have period, age and T$_{eff}$ measurements for $\sim$ 145 stars to which we need to fit a function of the following form:

\begin{equation}
P = P_ot^n
\end{equation}

where

\begin{equation}
P_o = P_o(T_{eff}) = a(T_{eff} - T_{max})^b,
\end{equation}

\begin{equation}
n = n(T_{eff}) = c(T_{eff} - T_{max})
\end{equation}

i.e.\ the current rotation period depends on time and effective temperature only (n.b.\ no longer separable!)

Uncertainties lie on all three parameters: period, age and T$_{eff}$ and are (at least assumed) Gaussian for P and T$_{eff}$ in linear space.
Uncertainties on age are Gaussian in log space, a feature of the asteroseismic age determination (they probably use a logarithmic grid, something like that).

Because the uncertainties are not all well described by a Gaussian I am adopting a total least-squares approach.
This involves drawing a number of posterior samples and marginalising over those samples, so the log likelihood for each sample is summed inside the main log likelihood summation.

This solves the problem of trying to fit a linear function with non-Gaussian uncertainties on all three dimensions, however this is not the whole story\ldots

Another problem is that there exists a cutoff temperature above which the above relation is no longer followed --- the \emph{Kraft break}.
Unfortunately we don't \emph{really} know where this cutoff is, and even if we did, there is obviously a finite probability of each data point lying to the right of the Kraft break*.
So, we can't just remove all stars that are more massive than 1.3 M$_{\odot}$ from our sample.
A question is --- do we want to \emph{learn} about where the Kraft break is from the data?
Or should we just assume that our data are not good enough for this kind of inference (this is probably true).
So let's say we are going to fix the Kraft break at a certain T$_{eff}$, maybe trying a few in a grid?
But we still know that each data point will have some probability mass to the right of the Kraft break, even if its maximum likelihood value lies to the left.
This, again wouldn't be a problem if we had lots of computing power or patience --- by drawing enough posterior samples that every star has at least one sample to the right of the Kraft break.

As if things weren't complicated enough already, we also need to decide how to deal with MS stars vs subgiants.
The gyrochronology relations only apply to MS stars.
Previously I have taken a cut in logg, although I should probably use a linear function of logg and T$_{eff}$, plus, of course, each star has a finite probability of being an MS star.

*BTW --- when I say `to the right' of the Kraft break, I mean a star is coolwards and therefore included in the sample.
Otherwise it lies to the left, obv.

Importance sampling.
Hierarchical inference.
Computing the marginalised likelihood.
Would be a straightforward problem, except the likelihood function is complicated.
You draw samples from the posteriors as an easy way to compute the marginalisation --- to perform the integral.
Want to model both population simultaneously with a mixture model.
Potentially using three mixture models --- the MS, low mass stars, the subgiants, and the high mass MS stars.
Could potentially model the subgiants with a model that genuinely describes their behavior.
The high mass MS stars might need to be modelled as a broad Gaussian since they don't spin down appreciably, so will just have the rotation periods they were born with.

\end{document}

