\documentclass[12pt,preprint]{aastex}
\usepackage{amsmath}
\usepackage{breqn}
\usepackage{cite,natbib}
\usepackage{epsfig}
\usepackage{cases}
\usepackage[section]{placeins}
\usepackage{graphicx, subfigure}

\begin{document}

\title{Calibrating Gyrochronology with Kepler Asteroseismic targets}

\author{Ruth Angus$^1$, Suzanne Aigrain$^1$, Amy McQuillan$^2$, Daniel Foreman-Mackey$^3$,  William, J. Chaplin$^4$, Tsevi Maseh$^2$}
\affil{$^1$Department of Physics, University of Oxford, OX1 3RH, UK}
\affil{$^2$School of Physics and Astronomy, Raymond and Beverly Sackler, Faculty of Exact Sciences, Tel Aviv University, 69978, Tel Aviv, Israel}
\affil{$^3$Centre for Cosmology and Particle Physics, New York University, New York, NY, USA}
\affil{$^4$School of Physics and Astronomy, University of Birmingham, Edgbaston, Birmingham, B15 2TT, UK}

\section{Abstract}
\label{abs}

Measuring ages for intermediate and low-mass stars on the main
sequence is challenging, but important for a wide range of studies,
from Galactic dynamics to stellar and planetary evolution.
Among the available methods, gyrochronology is a powerful one, because it requires knowledge of only the star�s mass (or colour) and its rotation period.
However, it is not well calibrated at late ages, and suffers from large uncertainties.
Asteroseismology provides relatively precise age measurements for some of the brightest stars observed by Kepler.
We measured the photometric rotation periods of 144 stars with asteroseismic age in order to calibrate the gyrochronology relation and improve upon current methods of measuring the ages of MS field stars.
We use advanced statistical methods to model the relationship between rotation period, age, and mass (or colour or effective temperature) while accounting for measurement uncertainties in all three quantities.
Our sample includes both main sequence stars and subgiants, and straddles the Kraft break (only main sequence stars cooler than the Kraft break are expected to follow gyrochronology relations); and this must be taken into account when modelling the data.
Once our method is applied to the extended sample of published rotation periods for stars with reliable mass and age estimates, it should enable us to estimate ages for any star with a measured period and mass (or temperature), along with associated uncertainties that reflect both measurement errors and the intrinsic scatter in the gyrochronology relations.

\section{Introduction}
\label{intro}

Stellar ages are notoriously difficult to measure for main sequence (MS) stars.
% Chromospheric activity is often used as an age indicator, as well as lithium depletion, but the evolution of these properties on the main-sequence is poorly understood, plus, high resolution spectra are required to measure either of these properties.
Isochrone placement is one of the most commonly used dating methods for field stars, however it is model-dependent and imprecise, often producing uncertainties of order 100\% or more.
Cluster ages measured by fitting isochrones to an ensemble of coeval stars can have uncertainties as low as 10\% \citep{Soderblom_2010}.
Unfortunately, because the majority of nearby clusters are young, there is a significant deficit of precisely measured ages for old stars and for this reason the current gyrochronology relations are poorly calibrated at late ages.
Photometric rotation periods of stars in the older Kepler clusters will provide excellent anchors for gyrochronology.
The Kepler cluster study \citep{Meibom2011} aims to measure rotation periods for stars in all four open clusters in the Kepler field of view.
The two younger clusters: NGC 6866 (0.5 Gyr) and NGC 6811 (1 Gyr) have been completed, however, the older clusters: NGC 6819 (2.5 Gyr) and NGC 6791 (9 Gyr) are more difficult targets.
Identifying cluster members is tricky since the field is crowded and Kepler pixels are large.
% How many pixels do the clusters cover?
The gyrochronology relations of \citet{Mamajek_2008} were calibrated using activity indicators in young cluster stars and a small number of field stars.
Chromospheric activity, usually measured from the strength of emission in the centers of the broad Ca II H \& K emission lines $R'_{HK}$, is driven by the magnetic dynamo and therefore correlated with rotation (Kraft 1967; Noyes et al. 1984; Montesinos et al. 2001).
The evolution of stellar activity is still poorly understood, especially for older stars and whilst the \citet{Mamajek_2008} relation improves upon the age-colour dependence of the \citet{Barnes_2007} relation, it remains poorly calibrated at late ages.
No existing gyrochronology models correctly describe the spin-down behaviour of stars for all masses and at all ages.

\subsection{Angular momentum loss in Main Sequence stars}
Stars lose angular momentum over their MS lifetimes via a magnetised wind that is constrained to rotate with the stellar surface out to the alfven radius \citep{Weber_1967}.
The strength of the magnetic field at the stellar surface, and therefore the rate of angular momentum loss, depends on the mass and rotation period of a star (CITATION).
Due to this dependence all F, G and K stars converge onto a unique mass-period-age relation after $\sim$ the age of the Hyades: 650 Myrs \citep{Irwin_2009} (MORE CITATIONS).
In theory, therefore, stellar age can be inferred from mass (or some proxy thereof) and rotation period measurements alone.
The gyrochronology relation of \citet{Barnes_2007} was determined empirically from observations of the Hyades and other young clusters and tested on binary stars.
It can be written:

\begin{equation}
P = A^n \times a(B-V-c)^b
\label{eq:Barnes2007_2}
\end{equation}

where P, A,  B and V are rotation period (in days), age (in Myr),  and B and V band colours respectively. The values of n, a, b and c are tabulated in \ref{tab:constants}
This relation was further calibrated by Mamajek and Hillenbrand (2008) and James (2010) with chromospheric activity measurements of field stars. %citation needed
Their relation takes the same form as Barnes (2007) with revised parameters shown in table \ref{tab:constants}.
Note that \citet{Mamajek_2008} treat c, the parameter setting the position of the discontinuity, as a free parameter, whereas \citet{Barnes_2007} fix it at 0.4.
As a result, the \citet{Barnes_2007} relation can be applied to hotter stars.

\begin{deluxetable}{lcc}
\label{tab:constants}
\tablewidth{0pc}
\tablecaption{Values of a, b, c \& n in \citet{Barnes_2007} and \citet{Mamajek_2008}}
\tablehead{
\colhead{Parameter}&
\colhead{Barnes (2007)}&
\colhead{Mamajek and Hillenbrand (2008)}}
\startdata
a & $0.7725 \pm 0.011$ & $0.407 \pm 0.021$ \\
b & $0.601 \pm 0.024$ & $0.325 \pm 0.024$ \\
c & $0.4$ & $0.495 \pm 0.010$ \\
n & $0.5189 \pm 0.0070$ & $0.566 \pm 0.008$ \\
\enddata
\end{deluxetable}

Gyrochronology relations of this kind can only be applied to F, G and K MS stars.
M dwarfs are fully convective and have a different magnetic dynamo (CITE AMY).
Their ages cannot be calculated via asteroseismology as they do not show high signal-to-noise Solar-like oscillations, so they do not appear in our sample.
Stars with masses $\lesssim 1.3 M_{\odot}$ have shallow convective zones - they are almost fully radiative - and, again, they have a different dynamo-driven magnetic field \citep{Kraft1967}.
These massive stars remain rapidly rotating throughout their brief MS lifetimes and are therefore not suitable gyrochronology targets.

\subsection{Asteroseismic ages}
Ages measured with asteroseismology can be very precise, with uncertainties as low as 10\% in some cases.
Unlike activity and rotation period diagnostics, the precision of asteroseismic ages are not, themselves age dependent.
Pressure wave oscillations produce periodic luminosity variations on timescales of $\sim$ 5 minutes for solar-like stars.
These oscillations can be detected in short-cadence Kepler data, and a fourier transform of the time-series shows a series of peaks, corresponding to each oscillation mode.
The mean large frequency separation between oscillation modes, $\Delta\nu$, is a fundamental asteroseismic observable and is related to the square-root of stellar density.
The series of peaks is modulated by a gaussian envelope, the maximum of which is another fundamental asteroseismic observable, $\nu_{max}$
The frequency of each oscillation mode depends on the integrated sound speed along the path through the star and measuring the frequency separation between oscillation modes yields an estimate of the stellar density.
When combined with spectroscopic observations and compared with theoretical stellar evolution models, measurements of the oscillation mode frequencies can yield stellar ages.
The asteroseismic properties of 505 stars published in Chaplin et al (2013) were calculated from measurements of the mean large frequency separation, $\Delta\nu$ and the maximum of the gaussian envelope, $\nu_{max}$
These two fundamental asteroseismic parameters can be used, via the scaling relations below, to derive stellar ages, masses and radii.

\begin{equation}
\frac{\Delta\nu}{\Delta\nu_{\odot}} \approx \sqrt{\frac{M/M_{\odot}}{R/R_{\odot}^3}}
\label{eq:delta_nu}
\end{equation}

\begin{equation}
\frac{\nu_{max}}{\nu_{max,\odot}} \approx \frac{M/M_{\odot}}{(R/R_{\odot})^2\sqrt{(T_{eff}/T_{eff,\odot})}}
\label{eq:delta_nu}
\end{equation}

Ages provided in \citet{Chaplin2013}, calculated from the mean large frequency separation and this scaling relation have uncertainties of  $\approx$ 35\%.
If the frequency of each oscillation mode is measured individually, one can build a density profile of the star and provide a tighter constraint on the evolutionary stage of the star.
Ages derived from individual oscillation mode measurements can have uncertainties as small as 10\% (CITATION), however this is a still a manual process and therefore takes time.
Although asteroseismology is a potentially precise dating method, it can only be applied to bright stars observed by missions like Kepler that show solar-like oscillations \citep{Chaplin_2011}.
It is therefore essential that we have a well calibrated dating method like gyrochronology which can be applied to any F, G or K star with a measurable rotation period.

% INDEPENDENT OF AGE
% SHOW 3D FIGURE

\section{Rotation Period Measurement}
\label{rotation_period_measurement}

The Kepler light curves of these 505 asteroseismic targets display quasi-periodic variations on timescales corresponding to rotational periods of the stars due to transiting star spots.
The auto-correlation function (ACF) method for stellar rotation period measurement was developed by McQuillan et al (2012).
As an alternative to the standard Fourier decomposition and least-squares fitting of
sinusoidal models \citep{Zechmeister}) autocorrelation is much better suited to signals that are not sinusoidal or strictly periodic. It is more effective at
 distinguishing a true signal from its harmonics.
 For a detailed description of the advantages of the ACF method, see \citet{McQuillan}.

An autocorrelation function describes the self-similarity of a light curve at a range of lag times.
The highest peak in the ACF (usually also the first peak) is centered on the rotation period of the star.
In our implementation the first two peaks in the ACF were identified and the central
value of the first peak was temporarily accepted as the period (unless
the second peak was higher than the first in which case {\it its}
central value was taken).
Note that an important advantage of the ACF method over a periodogram approach is in the differentiability of harmonic signals produced by multiple active regions on the stellar surface from the true periodic signal: these scenarios usually produce ACFs in which the second peak is higher than the first.
Subsequent peaks in the ACF lying within 10\% of
integer multiples of the period w�ere identified.
The final period measurement is calculated from the mean separation between peaks lying at integer multiples
and the error calculated from the distribution of central peak
values.
In cases where only one peak was present in the ACF, the
central peak value was kept as the period and the error measured from
the width of the peak.  Example ACFs are shown in figure \ref{fig:subfigures} at the end of this document.


We measured the rotation periods of the stars in our
asteroseismic target sample from Kepler quarters 3-16.
While some
light curves displayed high amplitude, regular flux variations
produced by star spots, others were dominated by random noise or instrumental
systematics.
To ensure that the periodicities measured were truly
representative of stellar rotation periods, we split the available
light curves into sections, or 'subsets' and computed an ACF for
each subset.
These subsets were: quarters 3-6, quarters 7-10, quarters
11-14 and individual quarters 3 -16.

The ACFs of light curves that did not display high amplitude, regular
variation were often populated by many small, unevenly spaced peaks. We
required the height of the selected peak to be greater than zero and significant with respect to the
immediately surrounding region of the ACF, i.e. for the relative
height of the peak to be greater than some value (in this case,
0.1). We also required that more than one peak lying at an integer multiple of the first be present in the ACF.
In cases where one or more of these criteria were not met, no
period was measured for that particular section of that star's light curve.

In order for the period measurement of a star to be deemed `reliable',
we stipulated that a period had to be successfully measured in at
least two thirds of the
data subsets.
We also required that the successful period
measurements lie within 15\% of the median period value, or a harmonic of that
median.
Of the 505 targets in the original sample, rotation periods of 144 were reliably measured using the above process.
An example light curve and ACF for KIC 322300 is shown in figure \ref{fig:lc} and quarter-by-quarter period measurements in figure \ref{fig:ind_qs}.
All but one of the quarters for this star produced an ACF with a significant initial peak and repeating subsequent peaks and of those, all period measurements lie within 15\% of the median value.

Long cadence PDC-MAP data were used throughout this analysis (\citet{Smith_2012}, \citet{Stumpe_2012}).
The PDC-MAP data are the product of an initial systematics removal process applied by the Kepler team,
 in which large-scale linear trends are removed in order to improve planet transit search and
modelling capability.
PDC-MAP data are not, however, optimised to preserve stellar
variability: signal is removed on timescales longer than ~ 30 days (CITATION).
Additionally, light curves still contain large systematic features,
such as the exponential decays that appear after telescope shutdowns, which affect period measurement precision.


\section{Gyrochronology Calibration}

505 stars with asteroseismically determined ages were published in Chaplin (2013).
We successfully measured rotation periods for 144 of these 505.
Each star has an effective temperature, T from multi-band photometry, a photometric rotation period, P and an asteroseismically derived age, A and surface gravity, log g (G).
Each of these properties has an associated uncertainty, assumed to be independent and Gaussian for T and P and log-normal for A and G.
% need a plot showing this!

The stars in our sample cover temperatures ranging from 5400 to 7000 K.
Gyrochronology is not a applicable to hot stars or subgiants.
Stars with effective temperatures above the Kraft-break, $T_{eff} \sim$ 6250 K, \citep{Kraft1967} do not have a thick convective envelope and therefore cannot support a strong magnetic dynamo, so do not spin down during their MS lifetimes.
Subgiants spin down rapidly due to angular momentum conservation as they expand, leaving the gyrochronology mass-period-age plane.
We cannot simply exclude hot stars and subgiants from our sample during the modelling process --- we \emph{have} to model all three populations at once.
There are two reasons for this: firstly, we don't know the exact location of the Kraft-break, so it has to be a free parameter,
and secondly, all stars have some probability mass lying in all three regimes due to their observational uncertainties.
The subgiant regime is bounded by a function of effective temperature, $T$ and log g, $G$.

The gyrochronology relations of \citet{Barnes_2007} and \citet{Mamajek_2008} take the form:

\begin{equation}
P = A^n \times a(B-V-c)^b,
\end{equation}

where n, a, b and c are constants, taking slightly different values in the two versions.
Our relation will be of the form:

\begin{equation}
	A = P^{\delta} \times a(T - T_k)^{\beta}
\label{eq:functional_form2}
\end{equation}

Where now A is the dependent variable, since we want to produce a predictive distribution for the age of a star, given estimates of T and P.
Since this equation is linear in log-space, we fit the logarithmic form of this equation to the low-mass, MS stars:

\begin{equation}
	\log{A} = \delta \log{P} + \alpha + \beta \log{(T - T_k)},
\label{eq:log}
\end{equation}

where $\alpha, \beta, T_K$ and $\delta$ are free parameters.
For the high-mass MS stars we assume that age is independent of period and $T_{eff}$, and model the population as a log-normal distribution with mean and variance as free parameters tobe marginalised over.
Subgiant ages \emph{are} a function of period and $T_{eff}$.

For now, lets just address the low-mass MS stars that obey the gyrochronology relation.
We have a set of noisy observations, $\hat{A}, \hat{P}, \hat{T}$ and $\hat{G}$ for each star.
These observational values have effectively been 'drawn' from a probability distribution centered on the 'true', hidden values, $A, P, T$ and $G$ with variance characterised by the uncertainties, $\sigma_A, \sigma_P, \sigma_T$ and $\sigma_G$.
The likelihood of the data given the model, marginalised over hidden variables, can be written as:

\begin{equation}
	p(\{\hat{A}_n,\hat{P}_n,\hat{T}_n,\hat{G}\}|\theta) =
	\prod_{n=1}^{N} \int p(\hat{A}_n,\hat{P}_n,\hat{T}_n,\hat{G}_n,A_n,P_n,T_n,G_n|\theta)
	{\rm d}A_n {\rm d}P_n {\rm d}T_n,{\rm d}G_n
\label{eq:fullL}
\end{equation}

Where $\theta = \alpha, \beta, \delta, T_K$. For the low-mass MS stars the joint probability can be factorised as:

\begin{equation}
  p(\hat{A}_n,\hat{P}_n,\hat{T}_n,A_n,P_n,T_n,G_n|\theta) =
  p(A_n,P_n,T_n,G_n|\theta) p(\hat{A}_n|A_n)
  p(\hat{P}_n|P_n) p(\hat{T}_n|T_n) p(\hat{G}_n|G_n),
\label{eq:jointprob}
\end{equation}

\begin{equation}
	\propto p(A_n|P_n,T_n,G_n,\theta) p(\hat{A}_n|A_n)
	p(\hat{P}_n|P_n)p(P_n) p(\hat{T}_n|T_n)p(T_n) p(\hat{G}_n|G_n)p(G_n) dA_n dP_n dT_n dG_n
\label{eq:jointprob}
\end{equation}

and the marginalised likelihood for a single star can be written as the sum of the individual star likelihoods over the three different regimes, k:

\begin{multline}
p(\hat{P}_n,\hat{A}_n,\hat{T}_n,\hat{G}_n|\theta,T_K)  = \\
\sum_{k=1}^3\int p(A_n,T_n,P_n,G_n|\theta,T_K)
p(\hat{A}_n|A_n) p(\hat{T}_n|T_n) p(\hat{P}_n|P_n) p(\hat{G}_n|G_n)
{\rm d}A_n {\rm d}T_n {\rm d}P_n {\rm d}G_n,
\end{multline}
\label{eq:L1}

where $p(A_n,T_n,P_n,G_n|\theta,T_K)$ is different in the three regimes, as listed in table \ref{tab:jprob}.

\begin{deluxetable}{lc}
\label{tab:jprob}
\tablewidth{0pc}
\tablecaption{Probability of hidden variables for the three populations.}
\tablehead{
\colhead{Regime}&
\colhead{Joint probability distribution}}
\startdata
Low mass, MS & $p(A_n,P_n,T_n,G_n|\theta) = p_1(T_n|T_K) p_1(G_n) p_1(P_n) p(A_n|T_n,P_n,\theta)$ \\
High mass, MS & $p(A_n,P_n,T_n,G_n|Y, V) = p(P_n) p(T_n) p(G_n) p(A_n|Y, V)$  \\
Subgiants & $p(A_n,P_n,T_n,G_n|Z, U) = p(T_n) p(G_n) p(P_n) p(A_n|Z, U)$ \\
\enddata
\end{deluxetable}

Where Y, V, Z and U are the means and variances of the hot star and subgiant populations respectively.
Right now I am fixing $T_K$ and only including two regimes: the cool MS stars and the hot MS stars.
This approach produces the recalibrated Gyrochronology relation shown in figure \ref{fig:results2} and the posterior distributions in figure \ref{fig:fixie_pd}.
Having $T_K$ as a free parameter introduces a problem: all stars will preferentially be modelled by the less constraining broad Gaussian distribution.
Stars lying above the Kraft break must incur a likelihood penalty or else $T_K$ will be pushed to lower and lower temperatures.
Alternatively, stars lying below the Kraft break could be given a likelihood `boost'.
We can get around this problem in two ways: either grid over $T_K$, or introduce another parameter, W, describing the variance of the gyrochronology relation.
I.e. we no longer describe the relation between temperature, age and period as being an infinitely narrow relationship --- instead we give it a little `width' and cool stars get that likelihood boost.
For now I am going to attempt to include the subgiants too.
This will involve, for now, just a cut in log g.
So, if a star has log g > 4, it will be an MS star, otherwise it will be a subgiant, regardless of temperature.
The subgiant population will be modelled, once again, as a log-normal distribution with mean, Z and variance, U.
Of course, this isn't quite right because there \emph{is} a relation between period, mass and age for the subgiants, in fact, it is an even tighter relationship than that of the MS stars.
We could choose to model the subgiants with an analytic expression for period, (age), given temperature and age (period), such as the one in \citet{van_Saders_2013}, however, we would like to remain as model \emph{independent} as possible throughout this process.
Our definition of 'MS' stars is currently anything with log g > 4 (this is also the approach taken by \citet{Garcia_2014}).
We could use a better motivated function of log g and temperature to categorise MS stars and subgiants.
Although the point at which a star turns off the MS is relatively well known and understood, (conventionally its hottest point on the HR diagram), the point at which stars stop following the gyrochronology spin-down relation is less well understood.
In fact, up to this point it has only been theorised \citep{van_Saders_2013}.
Once again, the parameters of the function that separates MS stars and subgiants should be free parameters.
We are going to come up against the same herdles as above: the likelihood functions above and below the MS turnoff must be properly normalised, or else the cut-off will 'want' to include all the MS stars too.
For the moment, therefore, I will be fixing those parameters.

Now that I seem to have a comparison version which does not treat the uncertainties correctly, that might be interesting to include?

Future:
still need to get period errorbars sorted.
remember - I also need to look at the list of binaries!
Consider C type stars too (less than 100 Myrs old) and ultra-slow rotators.
Consider giving the relation a 'breadth'. Then you need to think about Gibbs sampling or similar.
Consider adding in all existing data.
Consider gridding over different values of Tk and MS turnoff.
Ideally we want a predictive distribution for the age of a star, but does that mean we will get screwed by the age - mass dependence?
Do we also want a colour dependent relation?

When relaxing priors n wants to get bigger - this might just be because there isn't enough data (and priors should perhaps be stronger?).

\subsection{Modelling MS turn-off}
Modelling the MS turn-off requires some thought.
Figure \ref{fig:l_vs_t} shows log $g$ vs T$_{eff}$ for the entire sample of stars with the MS-subgiant boundary indicated with the black line.
This boundary was calculated using the Yonsei-Yale isochrones where we define the MS turnoff point as the hottest point on the stellar evolution track for a given mass.
The majority of our sample lies above the line: most of our stars are subgiants.
However that does not, necessarily mean that we can only model stars \emph{below} the line with the gyrochronology relation: stars continue to follow this relation for some time after they begin to leave the MS \citep{van_Saders_2013}.
So, just as with the Kraft break, the gyrochronological MS turn-off is a fuzzy line.

% \begin{figure}[ht]
% \begin{center}
% \includegraphics[width=6in, clip=true, trim=0 0 0.5in 0]{/Users/angusr/Python/Gyro/plots/l_vs_t.png}
% \caption{log $g$ vs $T_{eff}$ for all stars in our sample. The black line indicates the boundary between subgiants, which lie above the line and MS stars which lie below.}
% \label{fig:l_vs_t}
% \end{center}

\subsection{Priors}
When modelling sparse data with large uncertainties, the choice of prior is important as it will have a relatively large impact on the results.
A prior should be a true reflection of one's belief.
In this case, we know that the parameter, n is around 0.5.
Parameters a and b on the other hand are slightly less well defined.

% Include_subs.py has the subgiants as well
% A_free_Tk_fixed.py has Y and V as free parameters, but Tk is fixed.
% A_free.py has Y, V and Tk as free parameters.
% fixed_Tk.py has Suzanne's likelihood function with fixed Tk
% realdata.py has Suzanne's likelihood function with free Tk
% chisq.py is the dumbass minimum chisq code which does not take uncertainties into account properly

\begin{figure}[ht]
\begin{center}
\includegraphics[width=6in, clip=true, trim=0 0 0.5in 0]{/Users/angusr/Python/Gyro/plots/np_vs_a3.png}
\caption{Rotation period vs age for 58 MS stars with $M<1.4M_\odot$, coloured according to mass. % should be according to period!
Isomass lines were calculated using the relation in \citet{Mamajek_2008}}
\label{fig:results2}
\end{center}
\end{figure}

\begin{figure}[ht]
\begin{center}
\includegraphics[width=6in, clip=true, trim=0 0 0.5in 0]{/Users/angusr/Python/noisy-plane/old_figs/fixie_pd.png}
\caption{Posterior distributions}
\label{fig:fixie_pd}
\end{center}
\end{figure}

\subsection{Accounting for the `non-narrow' relationship}

Because we assume that the generative process for the data is an intrinsically noisy one, we need to add two extra parameters describing the mean and variance of a Gaussian perturbation to the plane.
In this the model takes the form:

We have assumed that we have been given data that parameterise posterior probability distributions with uninformative priors.
What are the implications if this is not the case?

\begin{equation}
	A_n = \alpha(T_n - T_K)^\beta \times P_n^\delta + E_n
\end{equation}

Where the $E_n$ are noise contributions drawn from a Gaussian with some mean, $\mu_{S,n}$ and variance, [$\sigma^2_{S,n} + S_n^2$], where $\sigma_{S,n}$ is the uncertainty of star n and $S_n$ is some unknown parameter, that characterises the intrinsic deviation of the hidden variables from the plane.

Rotation period vs age for the sub-giants is shown in figure \ref{fig:subgiants}. Models of sub-giant rotation evolution from \citet{van_Saders_2013} are shown for comparison.
Qualitatively speaking the observations show good agreement with the models.
The relationship between age and rotation shown in these models is, unlike the empirical gyrochronology relations of MS stars, physically motivated.
Rotational breaking of sub-giants is a simple consequence of angular momentum conservation during the expansion phase and therefore easy to model.

\begin{figure}[ht]
\begin{center}
\includegraphics[width=6in, clip=true, trim=0 0 0.5in 0]{/Users/angusr/Python/Gyro/plots/p_vs_t_orig.png}
\caption{Rotation period vs $T_{eff}$ for 58 MS stars with rotation period measurements, coloured according to age.
Isochrones were calculated using the relation in  \citet{Mamajek_2008}.}
\label{fig:results}
\end{center}
\end{figure}

\section{Discussion}

Figure \ref{fig:p_vs_m} shows rotation period as a function of asteroseismic mass for the MS stars.
In practise stellar ages are estimated by fitting gyrochronological isochrones (gyrochrones) to their rotation periods and masses (or colours).
Again, it is clear that the \citet{Barnes_2007} and \citet{Mamajek_2008} isochrones are over-predicting rotation periods.
Rotation period is not strongly dependent on age for the more massive stars which is why the two different age populations, shown in pink and blue in figure \ref{fig:p_vs_m} entirely overlap.
The age of star with mass between 1.1 and 1.3 $M_{\odot}$ and rotation period between 0 and 15 days estimated from gyrochrone fitting would suffer from degeneracy: the isochrones are close together in this region of parameter space.
More reliable ages are obtainable for low-mass, slowly rotating stars, where there is greater separation between gyrochrones, however these stars are under-represented by our sample.
The larger number of rapid vs slow rotators is a consequence of rotation period measurement bias: with more revolutions per quarter it is much easier to measure short rotation periods.
The larger number of high mass stars is an asteroseismic bias.

We have seen that the existing gyrochronology relations over-predict the rotation periods of low-mass stars.
This can be understood as a consequence of core-envelope decoupling.

\section{Conclusion}

\bibliographystyle{plainnat}
\bibliography{Gyro_paper}

\begin{figure}[ht]
\begin{center}
\includegraphics[width=6in, clip=true, trim=0 0 0.5in 0]{/Users/angusr/angusr/ACF/ACF2/PDCQ3_output/plots_acf/3223000_full.png}
\caption{PDC-MAP and 'processed' light curves for KIC 3223000 with the ACF.}
\label{fig:lc}
\end{center}
\end{figure}

\begin{figure}[ht]
\begin{center}
\includegraphics[width=6in, clip=true, trim=0 0 0.5in 0]{/Users/angusr/angusr/ACF/ind_qs_figs/3223000.pdf}
\caption{Period measurements for quarters 3 - 16 of star `KIC 7771282'. The blue solid line indicates the median value and the shaded blue region marks its 15\% margin. The blue dashed line with surrounding shaded area indicates one half of the median period measurement with 15\% margin.  More than two thirds of the period measurements were present and consistent to within 15\% of the median for this star, so it passed the selection process.}
\label{fig:ind_qs}
\end{center}
\end{figure}

\end{document}

