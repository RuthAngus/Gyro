\documentclass[10pt,preprint]{aastex}
\usepackage{amsmath}
\usepackage{breqn}
\usepackage{cite,natbib}
\usepackage{natbib}
\usepackage{epsfig}
\usepackage{cases}
\usepackage[section]{placeins}
\usepackage{graphicx, subfigure}
\usepackage{color}
\usepackage{amsmath}
\usepackage{float}
\floatplacement{figure}{H}
% \usepackage[nomarkers,figuresonly]{endfloat}

\newcommand{\logg}{log \emph{g}~}
\newcommand{\teff}{$T_{eff}~$}
\newcommand{\prot}{$P_{rot}~$}

\newcommand{\ah}{$\hat{A}_n$}
\newcommand{\ph}{$\hat{P}_n$}
\newcommand{\ch}{$\hat{C}_n$}
\newcommand{\gh}{$\hat{G}_n$}
\newcommand{\yh}{$\hat{Y}_n$}
\newcommand{\teffh}{$\hat{T}_n$}
\newcommand{\chit}{$\chi^2$}

\newcommand{\feh}{[Fe/H]}
\newcommand{\dd}{\ensuremath{\,\mathrm{d}}}

\begin{document}

\title{Calibrating Gyrochronology using Kepler Asteroseismic targets}

\author{Ruth Angus$^1$, Suzanne Aigrain$^1$, Amy McQuillan$^2$, Daniel Foreman-Mackey$^3$,  William, J. Chaplin$^4$, Tsevi Mazeh$^2$}
\affil{$^1$Department of Physics, University of Oxford, OX1 3RH, UK}
\affil{$^2$School of Physics and Astronomy, Raymond and Beverly Sackler, Faculty of Exact Sciences, Tel Aviv University, 69978, Tel Aviv, Israel}
\affil{$^3$Centre for Cosmology and Particle Physics, New York University, New York, NY, USA}
\affil{$^4$School of Physics and Astronomy, University of Birmingham, Edgbaston, Birmingham, B15 2TT, UK}

% Notes for paper:
% Want to provide an estimate of the amount of intrinsic uncertainty on an age calculated with our method.
% Are the rapid rotators in our sample members of the "C" sequence?
% talk about uncertainties: infer rotation period uncertainties?
% could infer colour uncertainties, but this would be very hard so we assign them some value that we feel is representative.

\begin{abstract}
\label{abs}

Measuring ages for intermediate and low-mass stars on the Main Sequence (MS) is challenging but important for a wide range of studies, from Galactic dynamics to stellar and planetary evolution.
Among the available methods, gyrochronology is a powerful one because it requires knowledge of only the star's mass (or effective temperature, or colour) and its rotation period.
However, it is not well calibrated at late ages and suffers from large uncertainties.
Asteroseismology provides age measurements for some of the brightest stars observed by Kepler.
% We measured the photometric rotation periods of 153 stars with asteroseismic ages in order to calibrate the gyrochronology relations and improve upon current methods of measuring the ages of MS field stars. %pchange
We use rotation periods measurements of 283 stars with asteroseismic ages and 237 cluster and field stars with precise age measurements to calibrate the gyrochronology relations.
Advanced statistical methods are used to model the relationship between rotation period, age, and colour while accounting for measurement uncertainties in all three quantities.
Our sample includes both MS stars and subgiants, and straddles the Kraft break (only MS stars cooler than the Kraft break, $\sim$ 6250 K, are expected to follow gyrochronology relations); and this must be taken into account when modelling the data.
Our method has been applied to the extended sample of published rotation periods for stars with reliable age measurements.
It enables us to estimate an age for any cool MS star with a measured period and colour, along with an associated uncertainty that reflects both measurement errors and the intrinsic scatter in the gyrochronology relations.
Uncertainties on stellar ages calculated with this new gyrochronology relation will typically be of order ...\%.
We find that this sample of asteroseismic stars does not fall neatly onto the mass-period-age plane---there is a significant level of intrinsic scatter.
It is currently unclear whether this scatter is a manifestation of noisy data or if stars cease to be well described by gyrochronology in general, past a certain age.
Uncertainties on the ages of this asteroseismic sample are of order 35\%---advancements in age measurement precision anticipated over the coming months will be key for our understanding of the applicability of gyrochronology.

\end{abstract}

\section{Introduction}
\label{intro}

\subsection{Dating methods for field stars}

Many fields of astronomy rely on stellar ages; the ages of planet hosts are necessary for planetary system evolution studies, for example.
Unfortunately, ages are notoriously difficult to measure for main sequence (MS) stars---observable stellar properties evolve slowly on the MS and even high precision measurements often cannot pin down a star's age to within 20\% \citep{Soderblom2010}.
The most reliable and precise age measurements available are for clusters, where isochrones can be fitted to an ensemble of coeval stars with a range of masses.
% These coeval populations provide a statistical ensemble of stars, with a range of masses, to which isochrones can be fitted.
% Cluster ages can also be measured, or independantly verified, from their dynamical properties.
Uncertainties on cluster ages are typically of order 10\%.
Isochronally derived ages for field stars are much less precise than cluster ages and often have uncertainties of order 50\% or more.
A few alternatives to isochrone fitting exist.
Asteroseismology can provide age measurements for bright Solar-like oscillators, with typical uncertainties ranging from 10-35\%.
Since the internal structure of stars evolve over time, asteroseismic measurements of stellar density yeild ages.
For field stars which do not show high-signal-to-noise oscillations, or for which high precision photometry is not available, empirically calibrated dating methods can be used.

% set up likelihood as if you're going to ignore uncertainties.
% Explain the fact that you can have samples falling in different regimes for the different stars.
% look for papers citing older ones.
% show fit for individual clusters.
% The sun would be hard to detect in our survey.
% don't worry about this too much.
% If there is an intrinsic spread in the relation and we only measure the lower envelope.
% repeat the fit with a simulation - cutting off rotation periods.

\subsection{Gyrochronology}

Mass loss via magnetised stellar winds is thought to cause magnetic breaking of MS stars \citep{Weber1967}.
The rate of rotational period decay was quantified by \citet{Skumanich1972}, who observed $P_{rot} \propto \mathrm{Age}^0.5$.
% \citet{Skumanich1972} quantified the decay of stellar activity and rotational period as proportional square-root of age.
Later, \citet{Noyes1984_2} added a mass dependence was added to the period-activity-age relation: more massive stars were observed to spin down more slowly.
This mass-dependent rotation period decay can be understood as stellar wind driven angular momentum loss.
In angular momentum loss models, a stellar wind is assumed to corotate with the stellar surface out to the Alfv\'{e}n radius \citep{Weber1967}, at which point point it decouples and angular momentum is lost from the star.
The strength of the magnetic field at the stellar surface, and therefore the rate of angular momentum loss, is inversely proportional to rotation period \citep{Kawaler1988}.
So although stellar populations are born with a range of rotation periods, the rapid rotators rapidly lose angular momentum and rotation periods converge onto a unique sequence.
The timescale for convergence is around the age of the Hyades: 650 Myrs (\citealt{Radick1987}, \citealt{Irwin2009}).
After this time rotation periods are \emph{independent} of their initial values.

Note that the type of angular momentum evolution explored here can only be applied to F, G and K MS stars.
M dwarf rotation periods evolve over extremely long timescales and they often have not converged onto the mass-period-age plane, even after several Gyrs.
Stars with effective temperatures $\gtrsim$ 6250 K have shallow convective zones - they are almost fully radiative - and, again, they have a different dynamo-driven magnetic field \citep{Kraft1967}.
These massive stars retain their initial rotation period throughout their brief MS lifetimes and are therefore not suitable gyrochronology targets.

The term `gyrochronology' was coined by \citet{Barnes2003} who calibrated the relation between period, colour and age of the form

\begin{equation}
P = A^n \times a(B-V-0.4)^b
\label{eq:Barnes2007_2}
\end{equation}

where $P$, $A$, $B$ and $V$ are rotation period (in days), age (in Myr) and B and V band magnitudes respectively and $a$, $b$, and $n$ are free parameters.
\citet{Barnes2007} used 8 young open clusters, aged between 30 and 650 Myrs to calibrate $f(B-V)$ and the Sun to calibrate $g(t)$.
A minimum \chit fitting approach was used---taking observational uncertainties on rotational period measurements into account, but not those for age and colour measurements.
Minimum \chit fitting also assumes uncertainties are Gaussian, which may not always be the case (although could be a reasonable approximation).
Best-fit values of $n$, $a$ and $b$ are tabulated in \ref{tab:constants}.

This relation was further calibrated by \citet{Mamajek2008} using updated rotation period and age measurements of stars in open clusters $\alpha$ Per \citep{Prosser1995}, Pleiades (\citealt{Prosser2005}; \citealt{Krishnamurthi1998}), M34 \citep{Meibom2011_M34}, and Hyades (\citealt{Radick1987}; \citealt{Prosser1995}; \citealt{Radick1995}; \citealt{Paulson2004}; Henry, priv. comm.).
The Sun was used to determine the rotation period-age relation.
\citet{Mamajek2008} only retained fits that reproduced the rotation period of the Sun to within 0.1 days.
This approach relies on the Copernican principle---they assume that the Sun is a typical rotator for its mass and age.
If this assumption is invalid (it is, at least, not well motivated) then their resulting gyrochronological relation will incorrectly predict ages of stars older than the Hyades.
Once again the gyrochronology relation was treated as a separable function of colour and age and obtained minimum \chit best fit values of $n$, $a$ and $b$.
Whereas \citet{Barnes2007} fixes the position of the `colour discontinuity' at 0.4, \citet{Mamajek2008} allow it to be a free parameter in their model, calling it `$c$'.
Their best-fit values are also tabulated in \ref{tab:constants}.

Open clusters are invaluable tools for calibrating gyrochronology.
Unfortunately, the majority of nearby clusters are young---there is a significant dearth of precisely measured ages for old stars and it is for this reason that the current gyrochronology relations are poorly calibrated at late ages.
The Kepler cluster study \citep{Meibom2011} aims to measure rotation periods for stars in all four open clusters in the Kepler field of view: NGC 6866 (0.5 Gyr), NGC 6811 (1.1 Gyr), NGC 6819 (2.5 Gyr) and NGC 6791 (9 Gyr).
So far, rotation periods for 71 stars in NGC 6811 have been measured and member identification for the other three clusters is underway.
Once completed, the older Kepler clusters will provide excellent anchors for gyrochronology.

\begin{deluxetable}{lccc}
\label{tab:constants}
\tablewidth{0pc}
\tablecaption{Values of a, b, c \& n in \citet{Barnes2007} and \citet{Mamajek2008} and this work.}
\tablehead{
\colhead{Parameter}&
\colhead{\citet{Barnes2007}}&
\colhead{\citet{Mamajek2008}}&
\colhead{This work}}
\startdata
a & $0.7725 \pm 0.011$ & $0.407 \pm 0.021$ & $0.619~^{+0.14}_{-0.067}$\\
b & $0.601 \pm 0.024$ & $0.325 \pm 0.024$ & $0.471~^{+0.016}_{-0.032}$\\
c & $0.4$ & $0.495 \pm 0.010$ & $0.45$ \\
n & $0.5189 \pm 0.0070$ & $0.566 \pm 0.008$ & $0.482~^{+0.013}_{0.030}$\\
\enddata
\end{deluxetable}

\subsection{Asteroseismic ages}
\label{sec:asteroseismic_targets}

Acoustic (pressure) mode oscillations produce periodic luminosity variations on timescales of $\sim$ 5 minutes in Solar-like stars.
These oscillations can be detected in short-cadence Kepler data---a fourier transform of the time series reveals a series of narrow peaks at discrete frequencies, overtones of high radial order, $n$.
Non-radial oscillations with angular degree, $l$ are also detectable in Solar-like stars.
The frequency spacing, (or large frequency separation, $\Delta\nu_{nl} = \nu_{n+1l}-\nu_{nl}$) between consecutive overtones, $n$, of the same degree, $l$, is a fundamental asteroseismic observable.
The series of peaks in the Fourier transform is modulated by a gaussian envelope, the maximum of which is another fundamental asteroseismic observable, $\nu_{\mathrm{max}}$.
These two fundamental asteroseismic parameters together with effective temperature can be used, via the scaling relations below, to derive stellar masses and radii.

\begin{equation}
\frac{\Delta\nu}{\Delta\nu_{\odot}} \approx \sqrt{\frac{M/M_{\odot}}{R/R_{\odot}^3}}
\label{eq:delta_nu}
\end{equation}

\begin{equation}
\frac{\nu_{\mathrm{max}}}{\nu_{\mathrm{max},\odot}} \approx \frac{M/M_{\odot}}{(R/R_{\odot})^2\sqrt{(T_{\mathrm{eff}}/T_{\mathrm{eff},\odot})}}
\label{eq:delta_nu}
\end{equation}

In particular, the mean large frequency separation, $\Delta\nu$ is proportional to the mean stellar density---since stellar density changes over the MS lifetime of a star, this parameter enables us to measure ages.

\citet{Chaplin2013} calculated the ages of 505 Kepler targets using the above scaling relations with estimates of $\Delta\nu$ and $\nu_{\mathrm{max}}$.
A grid-based approach was used to measure stellar properties: values of $\Delta\nu$ were calculated for each R and M on the grid and compared with the observed $\Delta\nu$.
Ages quoted in \citet{Chaplin2013} are the combined result of six different grid-based model pipelines.
The uncertainties on the ages reflect discrepancies between results obtained using different sets.
Two sets of effective temperatures were used: one was derived using an Infra-Red Flux Method (IRFM) calibration (\citealt{Casagrande2010}, \citealt{SilvaAguirre2012}) and the other from a recalibration of the SDSS griz filter KIC photometry by \citet{Pinsonneault2012} using Yale Rotating Stellar Evolution Code (YREC) models \citep{Demarque2004}.
We use the second set of temperatures since they are slightly more precise.
% test: is our analysis sensitive to this choice?

The asteroseismic ages in \citet{Chaplin2013} have typical uncertainties of $\sim$ 35\%; however, it will be possible to derive more precise ages for some of these stars.
By measuring the frequency of each oscillation mode individually, not just the mean large serparation, one can build up a density profile of the star and provide a tighter constraint on its age.
Ages derived from individual oscillation mode measurements can have uncertainties as small as 10\% (\citet{Brown1994}, \citet{SilvaAguirre2013}), however measuring frequencies for individual oscillation modes is a manual process and can only be applied in the highest signal-to-noise cases.
\citet{Chaplin2013} predict that around 150 of the 505 stars will be suitable for this individual oscillation mode treatment.
We obtained precise ages for 42 stars from \citet{Metcalfe2014}, modelled with the Asteroseismic Modeling Portal (AMP) and 23 from Silva Aguirre (private communication).
Precise effective temperatures and metallicities were obtained from \citet{Bruntt2012}.
% Of the 42 stars in \citep{Metcalfe2014}, we only integrate the `simple stars' (cool dwarfs) into our sample---we ignore the `F stars' and `mixed mode' (subgiant) stars as these are not expected to follow the simple gyrochronology relation.
% {\color{red} Shouldn't I include these to be consistent?}

% Notes - apparently asteroseismic ages are easier to derive for older stars!
% Probably don't want to include the youngest clusters since they have so much spread and haven't converged!
% We are calibrating the relation between period, age and colour---this is different to the relation between period, age and mass because
% colour-mass relation is different for different compositions and this adds a level of systematic uncertainty.

\subsection{Paper structure}

This paper is laid out as follows: .....

\section{Rotation Period Measurements}
\label{sec:rotation_period_measurement}

The Kepler light curves of the 505 asteroseismic targets display quasi-periodic variations on timescales corresponding to rotational periods of the stars.
Flux variations are produced by active regions on the stellar surface, rotating in and out of view.
In order to measure periodic signals in the light curves we used the auto-correlation function (ACF) method, developed by \citet{McQuillan2013}.
An autocorrelation function describes the self-similarity of a light curve at a range of lags and the highest peak in the ACF (usually also the first peak) is centered on the dominant periodic signal in the time series.
As an alternative to the standard Fourier decomposition and least-squares fitting of sinusoidal models \citep{Zechmeister}, autocorrelation is better suited to signals that are not sinusoidal or strictly periodic and is more effective at distinguishing a true signal from its harmonics and sub-harmonics.
For a more detailed description of the advantages of the ACF method over sine-fitting periodograms, see \citet{McQuillan}.

Long cadence PDC-MAP data were used throughout this analysis (\citet{Smith_2012}, \citet{Stumpe_2012}).
The PDC-MAP data are the product of an initial systematics removal process applied by the Kepler team, in which large-scale linear trends are removed in order to improve planet transit search and modelling capability.
PDC-MAP data are not, however, optimised to preserve stellar variability: periodic signals longer than around 30 days are reduced in power.
Additionally, light curves still contain large systematic features, such as the exponential decays that appear after telescope shutdowns, which affect period measurement precision.
The Astrophysically Robust Correction (ARC), a detrending method developed by \citet{Roberts2013}, is better at preserving astrophysical signals, however, it has not yet been applied to all Kepler quarters, so we opt to use PDC-MAP.
% {\color{red} So why are we still using it then?! - explain!}
% Note - I HAVE done some injection tests... could talk about these, although it would probably involve quite a lot more work...

In our implementation of the ACF method the first two peaks in the ACF were identified and the central value of the first peak was temporarily accepted as the period (unless the second peak was higher than the first in which case {\it its} central value was taken).

Note that an important advantage of the ACF method over a periodogram approach is in its ability to differentiate between harmonic signals produced by multiple active regions on the stellar surface from the true periodic signal: these scenarios usually produce ACFs in which the second peak is higher than the first.
Subsequent peaks in the ACF lying within 10\% of an integer multiple of the period were identified.
The final, reported period is calculated from the mean separation between peaks that lie at integer multiples of the period and the uncertainty on the period is calculated from the distribution of central peak values.
Measuring the mean peak separation removes a bias that can sometimes be introduced when a long-term, slowly decaying trend dominates the ACF at small periods.
If the first ACF peak is small when superimposed onto this trend, the period will be underestimated as the central peak value will be shifted towards smaller periods.
In cases where only one peak was present in the ACF, i.e. for stars with periods that are a significant fraction of the time coverage of a Kepler quarter, the central peak value was kept as the period and the uncertainty measured from the width of the peak.
Example ACFs for Kepler light curves are shown in figure \ref{fig:subfigures} at the end of this document.
An example light curve and ACF for KIC 322300 is shown in figure \ref{fig:lc} and quarter-by-quarter period measurements in figure \ref{fig:ind_qs}.
All but one of the quarters for this star produced an ACF with a significant initial peak and repeating subsequent peaks and, of those, all period measurements lie within 15\% of the median value.

We used Kepler quarters 3-16 to measure rotation periods for the stars in our sample.
While some light curves displayed high amplitude, regular flux variations produced by star spots, others were dominated by random noise or instrumental systematics.
To ensure that the periodicities measured were truly representative of stellar rotation periods, we computed an ACF for each Kepler quarter, individually, for each star and required that the resulting periods were mutually consistent.
We required that period measurements lie within 15\% of the median period value, or within 15\% of twice and half the median in order to catch aliases.
The maximum period, detectable in one Kepler quarter with our ACF method was just over 20 days.
% Why? 90 / 2 = 45, /2 ~ 20
In order to include stars with rotation periods longer than 20 days, we also calculated ACFs for quarters 3-6, 7-11 and 12-15, joined together.
Those stars with rotation periods consitent across all three years were examined by eye and included in our sample if a reasonable rotation period was measured.
The final, reported rotation periods were calculated from quarters 3-16, joined together.

Of the 505 targets in the original sample 122 rotation periods were reliably measured using the above process.
We added an additional 31 targets with rotation periods measured by \citet{McQuillan_2014}, increasing our sample to 153.
Of those 153, 13 of them have precise asteroseismic ages from AMP modelling.

\begin{figure}[ht]
\begin{center}
\includegraphics[width=6in, clip=true, trim=0 0 0.5in 0]{/Users/angusr/Python/Gyro/plots/logg_vs_t_paper.png}
\caption{\logg vs \teff for the 153 asteroseismic stars. Stars with \teff $>$ 6250 K are red and those with \logg $<$ 4.0 are blue. The triangular data points are those with precise ages. {\color{red}To do: add ASTEC isochrones}}
\label{fig:p_vs_a}
\end{center}
\end{figure}

\citet{Garcia2014} use a combination of wavelet transforms and ACFs to measure the surface rotation periods of Kepler asteroseismic stars.
They use a Morlet mother wavelet which is the convolution of a sinusoid and a Gaussian.
They require that at least four full rotation periods are present in the data
They use two sets of data: PDC-MAP and KADACS and two periodicity measurement methods: ACF and wavelet transform.
They require all four of these possible data and analysis method combinations to be consistent to within 20\%.
They then visually check the data.
compute a wavelet transform for each light curve and project it onto the period axis, reinforcing the true period and suppresing harmonic and sub-harmonic signals.
The details of this method are outlined in \citet{Mathur2014}.

\subsection{Observational data}

\citet{Chaplin2013} calculated the ages of 505 Kepler targets using the above scaling relations.
A grid-based approach was used to measure stellar properties: values of $\Delta\nu$ were calculated for each R and M on the grid and compared with the observed $\Delta\nu$.
Ages quoted in \citet{Chaplin2013} are the combined result of six different grid-based model pipelines.
The uncertainties on the ages reflect discrepancies between results obtained using different sets.
Two sets of effective temperatures were used: one was derived using an Infra-Red Flux Method (IRFM) calibration (\citealt{Casagrande2010}, \citealt{SilvaAguirre2012}) and the other from a recalibration of the SDSS griz filter KIC photometry by \citet{Pinsonneault2012} using Yale Rotating Stellar Evolution Code (YREC) models \citep{Demarque2004}.
We use the second set of temperatures since they are slightly more precise, however our analysis is relatively insensitive to this choice.
% test: is our analysis sensitive to this choice?

The asteroseismic ages in \citet{Chaplin2013} have typical uncertainties of $\sim$ 35\%; however, it will be possible to derive more precise ages for some of these stars.
By measuring the frequency of each oscillation mode individually, not just the mean large serparation, one can build up a density profile of the star and provide a tighter constraint on its age.
Ages derived from individual oscillation mode measurements can have uncertainties as small as 10\% (\citet{Brown1994}, \citet{SilvaAguirre2013}), however measuring frequencies for individual oscillation modes is a manual process and can only be applied in the highest signal-to-noise cases.
\citet{Chaplin2013} predict that around 150 of the 505 stars will be suitable for this individual oscillation mode treatment.
We obtained precise ages for 42 stars from \citet{Metcalfe2014}, modelled with the Asteroseismic Modeling Portal (AMP), with effective temperatures and metallicities from \citet{Bruntt2012}.
% Of the 42 stars in \citep{Metcalfe2014}, we only integrate the `simple stars' (cool dwarfs) into our sample---we ignore the `F stars' and `mixed mode' (subgiant) stars as these are not expected to follow the simple gyrochronology relation.
% {\color{red} Shouldn't I include these to be consistent?}

% Notes - apparently asteroseismic ages are easier to derive for older stars!
% We are calibrating the relation between period, age and colour---this is different to the relation between period, age and mass because
% colour-mass relation is different for different compositions and this adds a level of systematic uncertainty.
The asteroseismic sample covers a large range of ages (see figure \ref{fig:p_vs_a}), however it does not provide good mass coverage across the entire range.
There are very few stars with temperatures below 6000 K (B-V $\sim$ 0.55) and of the low mass stars, most of them are old (note---the paucity of massive, old stars is an effect of MS turn-off).
We therefore added 261 stars to our sample from young clusters Coma Berenices (0.5 Gyr), Praesepe (0.588 Gyr), the Hyades (0.625 Gyr) and NGC6811 (1.1 Gyr) (see table \ref{tab:clusters}).
We only added clusters with ages greater than 0.5 Gyrs as these younger clusters often have large populations of rapid rotators that have not yet converged onto the gyrochronology plane.

\begin{deluxetable}{lcccc}
\label{tab:clusters}
\tablewidth{0pc}
\tablecaption{Clusters and References: (1) \citet{Soderblom2009}, (2) \citet{Hartman2010}, (3) \citet{Jones1996}, (4) \citet{Meibom2011_M34}, (5) \citet{Dobbie2009}, (6) \citet{CollierCameron2009}, (7) \citet{Khalaj2013}, (8) \citet{Kovacs2014}, (9) \citet{Perryman1998}, (10) \citet{Radick1987}, (11) \citet{Janes2011}, (12) \citet{Meibom2011}. NGC 6811 g-r colours were converted to dereddened B-V with $E_{(B-V)}$ = 0.1. The age of M 34 reported in \citet{Jones1996} is 200-250 Myr.}
\tablehead{
\colhead{Cluster}&
\colhead{Age (Gyr)}&
\colhead{Number of stars}&
\colhead{Age ref}&
\colhead{Rotation period ref}}
\startdata
% Pleides & 0.1 $\pm$ 0.05 & 1 & 2 \\
% M 34 & 0.225 $\pm$ 0.025 & 3 & 4 \\
Coma Ber & 0.5 $\pm$ 0.1 & 28 & 5 & 6 \\
Praesepe & 0.588 $\pm$ 0.137 & 161 & 7 & 8 \\
Hyades & 0.625 $\pm$ 0.05 & 22 & 9 & 10\\
NGC 6811 & 1.1 $\pm$ 0.2 & 50 & 11 & 12 \\
\enddata
\end{deluxetable}

\begin{deluxetable}{lccc}
\label{tab:field_stars}
\tablewidth{0pc}
\tablecaption{Rotation periods and B-V colours for field stars with precise ages.
References: (1) \citet{Metcalfe2012}, (2) \citet{Henry2000}, (3) \citet{Moffet1979}, (4) \citet{Li2012}, (5) \citet{Petit2008}, (6) \citet{Mermilliod1986}, (7) \citet{Bouvier2010}, (8) \citet{Donahue1996}, (9) \citet{Cox2000}, (10) \citet{Bazot2012}, (11) \citet{Yildiz2007}, (12) \citet{Hallam1991}, (13) \citet{Dumusque2012}.}
\tablehead{
\colhead{ID}&
\colhead{age}&
\colhead{\prot}&
\colhead{B-V}}
\startdata
16 Cyg B & 6.4 $\pm$ 0.4$^1$ & 31.5 $\pm$ 6.5$^2$ & 0.66 $\pm$ 0.01$^3$ \\
18 Sco & 3.66 $\pm$ 0.2$^4$ & 22.7 $\pm$ 0.5$^5$ & 0.64 $\pm$ 0.01$^6$ \\
The Sun & 4.568 $\pm$ 0.001$^7$ & 26.09 $\pm$ 0.1$^8$ & 0.65 $\pm$ 0.001$^9$ \\
$\alpha$ Cen A & 6 $\pm$ 1$^{10,11}$ & 28.8 $\pm$ 2.5$^{12}$ & 0.69 $\pm$ 0.01$^6$ \\
$\alpha$ Cen B & 6 $\pm$ 1$^{10,11}$ & 38.7 $\pm$ 5.0$^{13}$ & 0.90 $\pm$ 0.01$^6$ \\
\enddata
\end{deluxetable}

\begin{figure}[ht]
\begin{center}
	\subfigure[$P_{rot}$ vs \teff]{
            \label{fig:p_vs_t}
	    \includegraphics[width=3in, clip=true, trim=0 0 0.5in 0]{/Users/angusr/Python/Gyro/plots/p_vs_t_paper.png}
        }
	\subfigure[$P_{rot}$ vs B-V colour]{
            \label{fig:p_vs_bv}
	    \includegraphics[width=3in, clip=true, trim=0 0 0.5in 0]{/Users/angusr/Python/Gyro/plots/p_vs_bv_paper.png}
        }
    \end{center}
    \caption{ Photometric rotation period vs effective temperature and B-V colour for 153 Kepler asteroseismic targets. \teff was converted to B-V using \citet{Sekigchi2000}.
     }
   \label{fig:subfigures}
\end{figure}

Rotation periods and B-V colours for the Hyades were obtained from \citet{Radick1987} and an age of 650 Myrs was adopted from \citet{Perryman1998}.
NGC 6811 is a cluster in the Kepler field with an age of 1.1 $\pm$ 0.2 Gyr \citep{Janes2011} with photometric rotation periods for cluster members measured by \citet{Meibom2011}.
These stars have g-r colours which were converted to dereddened B-V with $E_{(B-V)} = 0.1$.
Rotation periods for the 590 Myr \citep{Khalaj2013} cluster Praesepe were published in \citet{Kovacs2014}, with dereddened B and V colours from the APASS database ((http://www.aavso.org/apass).

A further 5 field stars with precise age measurements were added to the sample: 16 Cyg B, Alpha Cen A and B, 18 Sco and, of course, the Sun (see table \ref{tab:field_stars}).
An asteroseismic age for 16 Cyg B was obtained from \citet{Metcalfe2012}, $T_{eff}$ from \citet{Ramirez2009} and rotation period from \citet{Henry2000} with B and V colours from \citet{Moffett1979}.
An age of 6 $\pm$ 1 Gyr was adopted for Alpha Cen AB, based on the analysis by \citet{Bazot2012} and \citet{Yildiz2007} (note that ages derived for Alpha Cen AB are extremely model dependent).
B and V colours were obtained from \citet{Mermilliod1986} and rotation periods from \citet{Hallam1991} and \citet{Dumusque2012} for A and B, respectively.
An age of 4.568 $\pm$ 0.001 Gyr for the Sun was taken from \citet{Bouvier2010}, and a latitudinal mean rotation period, observed by \citet{Donahue1996}.
% We also include the Sun as an old anchor datum, adopting a period of 26.09 days which is the latitudinal mean observed by Donahue et al. (1996) (the solar rotation ranges from �25 days near the equator to �32 days near the poles).
The age of 3.66 $\pm$ 0.2 for 18 Sco was taken from \citet{Li2012}, with rotation period from \citet{Petit2008} and B and V colours from \citet{Mermilliod1986}.

Despite having effective temperatures for the asteroseismic targets, we converted $T_{eff}$ to B-V colours, using the relation in \citet{Sekiguchi2000}---see figure \ref{fig:subfigures}.
This conversion is imprecise since the metallicites of the asteroseismic targets are the field average and not calculated per star.
Eventually we intend to provide an effective temperature calibration in addition to this colour calibration.
% however we concluded that converting $T_{eff}$ to colour would be more efficient than converting cluster star colours to $T_{eff}$, since we have more information about the asteroseismic sample.

The entire set of 418 stars is shown in figure \ref{fig:3d}. Asteroseismic targets are shown in black, with supplementary cluster and field stars in red.

\begin{figure}[ht]
\begin{center}
\includegraphics[width=6in, clip=true, trim=0 0 0.5in 0]{/Users/angusr/Python/Gyro/plots/3d_angled.png}
\caption{Colour, age and rotation periods of all 418 stars. Asteroseismic stars are black and additional cluster and field stars are red. {\color{red}add precise stars.}}
\label{fig:3d}
\end{center}
\end{figure}

\begin{figure}[ht]
\begin{center}
\includegraphics[width=6in, clip=true, trim=0 0 0.5in 0]{/Users/angusr/Python/Gyro/plots/p_vs_bv_paper2.png}
\caption{Photometric rotation period vs B-V colour for 153 Kepler targets (black) plus cluster and field stars (red). The blue stars have precise ages.}
\label{fig:3d}
\end{center}
\end{figure}

\begin{figure}[ht]
\begin{center}
\includegraphics[width=6in, clip=true, trim=0 0 0.5in 0]{/Users/angusr/Python/Gyro/plots/p_vs_a_paper2.png}
\caption{Photometric rotation period vs age for 153 Kepler targets (black) plus cluster and field stars (red). The blue stars have precise ages. {\color{red} add 7 other precise stars.}}
\label{fig:p_vs_a}
\end{center}
\end{figure}

\section{Calibrating the Gyrochronology relation}
\label{sec:gyro_cal}

\subsection{The model}

The 153 asteroseismic stars in our sample have B-V colours converted from effective temperatures, photometric rotation periods, P and asteroseismically derived ages, A and surface gravities, \logg (G).
Each measurement of these properties is assumed to be independent and has an associated uncertainty, assumed to be Gaussian.
The assumption of independence breaks down in the case where there is significant systematic bias in the rotation period, colour or age measurement methods, however we do not expect very large biases, so we do not expect this to be a problem.
The 265 cluster and field stars added to our sample do not have \logg values; however, since we only use \logg to separate the populations of subgiants and dwarfs (and we assume that the cluster and field stars are dwarfs) this shouldn't hurt our analysis.

Hot stars and subgiants follow a different gyrochronology relation to MS dwarfs: stars with effective temperatures above the Kraft-break, $T_{eff} \sim$ 6250 K, \citep{Kraft1967} do not have a thick convective envelope and cannot support a strong magnetic dynamo, so do not spin down appreciably during their MS lifetimes.
Subgiants spin down rapidly as they expand, due to angular momentum conservation, and thus diverge from the gyrochronological mass-period-age plane.
The point in their evolution at which they turn off the `gyrochronological MS turnoff' is difficult to define.
Classically, MS turnoff is defined as the hottest point on a star's path on the HR diagram (before it ascends the giant branch) but theory predicts that evolving stars begin the process of spinning down relatively slowly after leaving the `classically defined' MS \citep{vanSaders2013}.
For this reason we choose a very simple definition of MS turnoff---we use a \logg cut of 4.0 to differentiate dwarfs from giants.
We do not simply exclude hot stars and subgiants from our sample during the modelling process---we model all three populations simultaneously.
This allows for the fact that stars have some probability mass lying in all three regimes due to their large observational uncertainties.
In addition to hot stars and subgiants, there is further population of contaminating stars in our sample: rapid rotators.
Around 20 stars with rotation periods of less than $\sim$ 5 days do not lie on the standard gyrochronology mass-period-age plane.
These stars could plausibly be synchronised binaries or stars with unseen, close-in, massive planets (\citet{Poppenhaeger2014}, \citet{Beky2014})
{\color{red} check to see if this population still exists and potentially take this out}

We model four stellar populations simultaneously: cool MS stars, hot MS stars, subgiants and rapid rotators.
The hot MS stars are defined as those with B-V $<$ 0.45, corresponding to \teff $\approx$ 6250 K for Solar metallicity and \logg.
Since there is no dependence of age on rotation period for massive MS stars, their ages are modelled as a log-normal distribution with mean and standard deviation, Y and V, free parameters.
Subgiant ages \emph{do} depend on period and $T_{eff}$; however, since we are not interested in the rotational properties of these stars for the purposes of gyrochronology calibration, we also model them with a log-normal distribution with mean and standard deviation, Z and U, free parameters.
% We could model the subgiants with an analytic expression for age, given colour and period, such as the one in \citet{vanSaders2013}, however, we would like to remain as model \emph{independent} as possible throughout this process.
We use a mixture model for the remaining two populations of cool MS stars: those that follow the gyrochronology relation and those with unusually fast rotation periods.
The fast rotators are treated as 'outliers' and modelled with another log normal distribution with mean and standard deviation, X and U, again inferred from the data.
In our mixture model an additional parameter, $P_b$, is the probability of each data point being an outlier.

Ideally both the hot star (B-V $<$ 0.45) and subgiant (\logg $<$ 4) boundaries would be free parameters in our model.
However, since these two populations are modelled with a relatively unconstraining log-normal distribution, these boundary parameters would not be well behaved.
Both would be pushed to higher and higher values until all stars were modelled with a log normal distribution.
In order to avoid this problem, we fixed these two boundaries.
A future analysis could deal with this issue by maximising likelihood over a grid of boundary parameter values.
% {\color{red}Try using different values and seeing how sensitive you are!}
Alternatively, one could avoid the assumption that the gyrochronology relation is infinitely narrow and assign it some intrinsic width, which would also be a free parameter.

Our likelihood function for the cool dwarf regime can be written as follows:

% \begin{eqnarray}
% 	\mathcal{L}_{gyro} = \prod_{i=1}^N \left[ \frac{1-P_b}{\sqrt{2\pi\sigma_{i}^2}}
% 	\exp\left({-\frac{\left(\sum_{j=1}^J [A_{i}- (P_{ij} \times \frac{1}{a}([B-V]_{ij}-c)^{-b})^{\frac{1}{n}}]\right)^2} {2\sigma_{i}^2}} \right)  \right] \nonumber \\
% 	+ \frac{P_b}{\sqrt{2\pi[U+\sigma_{i}^2]}} \exp \left( -\frac{[A_i - X]^2}{2[U+\sigma_{i}^2]}  \right)
% \end{eqnarray}
% \label{eq:likelihood}

\begin{eqnarray}
	\mathcal{L}_{gyro} = \prod_{i=1}^N \left[ \frac{1-P_b}{\sqrt{2\pi\sigma_{i}^2}}
	\exp\left({-\frac{\left[P_i - A_i^n \times a(B_i-V_i-c)^b\right]^2}{2\sigma_{i}^2}} \right)  \right] \nonumber \\
	+ \frac{P_b}{\sqrt{2\pi[U+\sigma_{i}^2]}} \exp \left( -\frac{[A_i - X]^2}{2[U+\sigma_{i}^2]}  \right)
\end{eqnarray}
\label{eq:likelihood}

where $P_b$ is the probability that a star is drawn from the background log-normal distribution, with mean, X and standard deviation, U.

The likelihood function for the hot and subgiant regimes can be written

\begin{eqnarray}
	\mathcal{L}_{hot} = \prod_{i=1}^N \left[ \frac{1}{\sqrt{2\pi[V+\sigma_{i}^2]}}
	\exp\left({-\frac{\left(A_{i}- Y\right)^2} {2[V+\sigma_{i}^2]}} \right)  \right] \\
	\nonumber \\
	\mathcal{L}_{subgiant} = \prod_{i=1}^N \left[ \frac{1}{\sqrt{2\pi[W+\sigma_{i}^2]}}
	\exp\left({-\frac{\left(A_{i}- Z\right)^2} {2[W+\sigma_{i}^2]}} \right)  \right] \nonumber \\
\end{eqnarray}

{\color{red} why should the Ns be different? It's the same N, surely.}

respectively, where Y and Z are the respective means and V and W the respective standard deviations.
The likelihood can be written as the sum of likelihoods over the three different regimes:%, k:

\begin{equation}
	\mathcal{L} = \mathcal{L}_{gyro} + \mathcal{L}_{hot} + \mathcal{L}_{sub}
\end{equation}

\subsection{Accounting for observational uncertainties}

We postulate that there is a deterministic relationship between the "true" rotation period, $P_n$, of a star and its "true" age, $A_n$, and colour, $C_n$, described by equation \ref{eq:plane}.
By "true" we mean the value an observable property would take, given infinitely high signal-to-noise measurements.
$P_n$ also depends on \logg($G_n$) since this property determines whether a star falls in the dwarf or subgiant regime.

We would like to sample the posterior probability of the parameter vector $\theta = \{a, b, n, X, Y, Z, U, V, W, P_b\}$ conditioned on a set of noisy observations \ph, \ah, \ch, and \gh and therefore need to compute the marginalised likelihood,

\begin{equation}
	p(\{\hat{P}_n,\hat{A}_n,\hat{C}_n,\hat{G}\}|\theta) =
	\prod_{n=1}^{n} \int p(\hat{P}_n,\hat{A}_n,\hat{C}_n,\hat{G}_n,P_n,A_n,C_n,G_n|\theta)
	{\rm d}P_n {\rm d}A_n {\rm d}C_n,{\rm d}G_n
\label{eq:fulll}
\end{equation}

The joint probability can be factorised as

\begin{align}
	p(\hat{P}_n,\hat{A}_n,\hat{C}_n,\hat{G}_n,P_n,A_n,C_n,G_n\,|\,\theta) = & \nonumber \\
	p(P_n)\,p(C_n)\,p(G_n)\,p(P_n\,| & \,A_n,C_n,G_n,\theta)\
        p(\hat{P}_n\,|\,P_n)\,p(\hat{A}_n\,|\,A_n)\,p(\hat{C}_n\,|\,C_n)\,p(\hat{G}_n\,|\,G_n)
\nonumber
\end{align}

where, in the cool dwarf regime
\begin{eqnarray}
p(P_n\,|\,A_n,C_n,G_n,\theta) =
	& (1-P_b)~\delta \left [P_n - \left(\left[A^n \times a(B-V - c)^b\right]^n\right) \right] \quad \\
	& +~P_b~\times \left(\sqrt{2\pi[U^2+\sigma^2]}\right)^{-1/2}~\exp\left({\frac{(P_n-X)^2}{2[U^2+\sigma^2]}}\right),
\end{eqnarray}
in the hot dwarf regime
\begin{eqnarray}
p(P_n\,|\,A_n,C_n,G_n,\theta) = \left(\sqrt{2\pi[V^2+\sigma^2]}\right)^{-1/2}~\exp\left({\frac{(P_n-Y)^2}{2[V^2+\sigma^2]}}\right)
\end{eqnarray}
and in the subgiant regime
\begin{eqnarray}
p(P_n\,|\,A_n,C_n,G_n,\theta) = \left(\sqrt{2\pi[W^2+\sigma^2]}\right)^{-1/2}~\exp\left({\frac{(P_n-Z)^2}{2[W^2+\sigma^2]}}\right),
\end{eqnarray}

We can compute equation \ref{eq:fulll}, up to an unimportant constant, using a sampling approximation.
The values of \ah, \ch~ (or \teffh) and \gh with uncertainties, $\sigma_A$, $\sigma_C$ and $\sigma_G$, reported in catalogues provide constraints on the posterior probability of those variables, under a choice of prior.
Ideally, these catalogues would provide samples from their posterior PDFs for the asteroseismically determined parameters \ah, \teffh and \gh which we could use directly.
i.e. samples from
\begin{equation}
p(\hat{A}_n, \hat{C}_n, \hat{G}_n|D_n) = \frac{1}{\beta_n}p(D_n|\hat{A}_n, \hat{C}_n, \hat{G}_n)p_0(\hat{A}_n, \hat{C}_n, \hat{G}_n)
\end{equation}
where $p(D_n|\hat{A}_n, \hat{C}_n, \hat{G}_n)$ is the likelihood of the data, $D_n$ (in this case, the set of Kepler lightcurves and supplementary \teff and \feh values), $\beta_n$ is a normalisation constant and $p_0(\hat{A}_n, \hat{C}_n, \hat{G}_n)$ is an uninformative prior PDF, chosen by the fitter (\citet{Chaplin2013} use a flat prior in age and log $g$).

In the absence of posterior PDF samples we generate our own from Gaussian distributions with means, \ah, \ch, \gh and standard deviations, $\sigma_A^2$, $\sigma_C^2$ and $\sigma_G^2$.
We generate $J_n$ posterior samples:
\begin{eqnarray}
A_n^{(j)} &\sim& p(A_n\,|\,\hat{a}_n) \nonumber \\
C_n^{(j)} &\sim& p(C_n\,|\,\hat{C}_n) \nonumber \\
G_n^{(j)} &\sim& p(G_n\,|\,\hat{G}_n)
\end{eqnarray}
and use these to evaluate $p(\hat{P_n}|P_n)$ up to a normalisation constant.
We then evaluate the marginalized likelihood for a single star as follows

\begin{align}
	p(\hat{P}_n,\hat{A}_n,\hat{C}_n,\hat{G_n}\,|\,\theta) = & \nonumber\\
\int
p(P_n)\,p(C_n)\, & p(G_n)\,p(P_n\,|\,A_n,C_n,G_n,\theta)\,
	p(\hat{P}_n\,|\,P_n)\,p(\hat{A}_n\,|\,A_n)\,p(\hat{C}_n\,|\,C_n),p(\hat{G}_n\,|\,G_n)
    \dd P_n \dd A_n \dd C_n \dd G_n \nonumber\\
&\propto \int
    p(P_n\,|\,A_n,C_n,G_n,\theta)\,p(\hat{P}_n\,|\,P_n)\,
    p(A_n\,|\,\hat{A}_n)\,p(C_n\,|\,\hat{C}_n),p(G_n\,|\,\hat{G}_n)
    \dd P_n \dd A_n \dd C_n \dd G_n \nonumber\\
&\approx \frac{1}{J_n} \sum_{j=1}^{J_n}p(\hat{P}_n\,|\,P_n^{(j)})
\end{align}

where $P_n^{(j)}$ is computed from the posterior samples.
Finally, the full marginalized log-likelihood is
\begin{eqnarray}
	\log p(\{\hat{P}_n,\hat{A}_n,\hat{C}_n,\hat{G}_n\}\,|\,\theta) &\approx&
    \log\zeta + \sum_{n=1}^N
        \log \left[ \sum_{j=1}^{J_n}p(\hat{P}_n\,|\,P_n^{(j)}) \right ]
\end{eqnarray}
where $\zeta$ is an irrelevant normalization constant and $p(\hat{P}_n|P_n)$ is defined in \ref{eq:likelihood}.
We used {\tt emcee} \citep{Foreman-Mackey2013}, an affine invariant, ensemble sampler Markov Chain Monte Carlo (MCMC) algorithm, to explore the posterior probability distributions of the model parameters, $\theta$.


\section{Results and Discussion}
\label{sec:results}

Marginal posterior distributions for all model parameters are shown in figure \ref{fig:triangle_full}.
% Figure ... shows the posterior distributions for the parameters of interest only, having marginalised over the nuisance parameters describing the log-normal distributions of subgiants, hot stars and rapid rotators.
% The full set of posterior samples can be downloaded at (URL...).
Our resulting Maximum \emph{a posteriori} values of the three gyrochronology parameters, with their 16th and 84th percentile values are tabulated in \ref{tab:results}.
Figures \ref{fig:650} - \ref{fig:13gyr} show the new gyrochronology relation in period-colour space over a sequence of ages.
The shaded region shows the 16th and 84th percentile uncertainties of the new relation.
The stars plotted have ages that fall within 1 $\sigma$ of the reference age.

% % comment this out for now - slows down compile time cos it's so biiiig
\begin{figure}[ht]
\begin{center}
\includegraphics[width=6in, clip=true, trim=0 0 0.5in 0]{/Users/angusr/Python/noisy-plane/triangle45_2.png}
\caption{Marginalised posterior probability distributions of all parameters in our model. The blue lines are centred on the maximum \emph{a posteriori} parameter values. This figure was made using triangle.py \citep{Foreman-Mackey_triangle}. {\color{red}Note - this is just a placeholder and not the final plot!}}
\label{fig:triangle_full}
\end{center}
\end{figure}

% Figures \ref{fig:p_vs_bv_solar} and \ref{fig:p_vs_a_solar} show rotation period vs colour and age for stars that lie within 2 $\sigma$ of the Sun's colour and age: 0.65 and 4.568 Gyr.
Figure \ref{fig:p_vs_bv_solar} shows rotation period vs colour for stars with age within 1 $\sigma$ of the Sun's: 4.568 Gyr.
Figure \ref{fig:p_vs_a_solar} shows rotation period vs age for stars with B-V colour within 10\% of the Sun's: 0.65.
Our new gyrochronology relation falls below both those of \citet{Barnes2007} and \citet{Mamajek2008} which seem to overpredict the rotation periods of the asteroseismic stars.
It is clear that we do not have many Solar analogues in our sample: most stars of a similar age are hotter.
{\color{red} Note - I don't want to interpret too much before we have the final results!}

\begin{figure}[ht]
\begin{center}
	\subfigure[0.65 Gyr (age of the Hyades)]{
            \label{fig:650}
	    \includegraphics[width=2in, clip=true, trim=0 0 0.5in 0]{/Users/angusr/Python/Gyro/plots/p_vs_bv0.png}
        }
	\subfigure[1 Gyr (age of NGC 6811)]{
            \label{fig:1gyr}
	    \includegraphics[width=2in, clip=true, trim=0 0 0.5in 0]{/Users/angusr/Python/Gyro/plots/p_vs_bv1.png}
        }
	\subfigure[2 Gyr]{
            \label{fig:2gyr}
	    \includegraphics[width=2in, clip=true, trim=0 0 0.5in 0]{/Users/angusr/Python/Gyro/plots/p_vs_bv2.png}
        }
	\subfigure[3 Gyr]{
            \label{fig:3gyr}
	    \includegraphics[width=2in, clip=true, trim=0 0 0.5in 0]{/Users/angusr/Python/Gyro/plots/p_vs_bv3.png}
        }
	\subfigure[4.568 Gyr (Solar age)]{
            \label{fig:sungyr}
	    \includegraphics[width=2in, clip=true, trim=0 0 0.5in 0]{/Users/angusr/Python/Gyro/plots/p_vs_bv4.png}
        }
	\subfigure[6 Gyr]{
            \label{fig:6gyr}
	    \includegraphics[width=2in, clip=true, trim=0 0 0.5in 0]{/Users/angusr/Python/Gyro/plots/p_vs_bv5.png}
        }
	\subfigure[8 Gyr]{
            \label{fig:8gyr}
	    \includegraphics[width=2in, clip=true, trim=0 0 0.5in 0]{/Users/angusr/Python/Gyro/plots/p_vs_bv6.png}
        }
	\subfigure[10 Gyr]{
            \label{fig:13gyr}
	    \includegraphics[width=2in, clip=true, trim=0 0 0.5in 0]{/Users/angusr/Python/Gyro/plots/p_vs_bv7.png}
        }
    \end{center}
    \caption{ \prot vs B-V colour for dwarfs within 1$\sigma$ of the reference age with the new gyrochronology relation and \citet{Barnes2007}, and \citet{Mamajek2008} for comparison. Asteroseismic targets are circles and cluster and field stars are red triangles. The shaded region represents the 16th and 84th percentile uncertainties.
     }
   \label{fig:subfigures2}
\end{figure}

\begin{figure}[ht]
\begin{center}
\includegraphics[width=6in, clip=true, trim=0 0 0.5in 0]{/Users/angusr/Python/Gyro/plots/p_vs_bv_solar.png}
\caption{Rotation period vs B-V colour for dwarfs with age within 1$\sigma$ of the Sun's age, 4.568 Gyr. The grey stars are hotter than 6250 K. The Sun is shown as a large grey circle.}
\label{fig:p_vs_bv_solar}
\end{center}
\end{figure}
%
\begin{figure}[ht]
\begin{center}
\includegraphics[width=6in, clip=true, trim=0 0 0.5in 0]{/Users/angusr/Python/Gyro/plots/p_vs_a_solar.png}
\caption{Rotation period vs age for cool dwarfs with colour within 10\% of the Sun's: 0.65, with gyrochronology relations of \citet{Barnes2007}, \citet{Mamajek2008} and this work. The shaded region represents the 16th and 84th percentile uncertainties. The Sun is shown as a large grey circle.}
\label{fig:p_vs_a_solar}
\end{center}
\end{figure}

There are a few potential explanations for the differences between previous and newly calibrated relations.
The first, and most obvious, is simply that the Sun is not a typical rotator for its age - it is slowly rotating.
Previous calibrations draw their age dependence almost exlusively from the Sun, assuming that it is representative of other stars with the same rotation period and mass.
In light of these new data, it may be time to reassess the `Sun is a representative star' viewpoint.
Another plausible explanation is that the asteroseismic ages are systematically too old; this would make the Sun appear to be slowly rotating for its age.
Alternatively, the ACF method could be underestimating rotation periods.
{\color{red} Do we have any reason to believe either of these to be true?}
% could test this

% NON-NARROW RELATIONSHIP

The goal of gyrochronology in general is to provide a means of predicting the age of a star, given observations of its \teff, or colour, and rotation period.
The `narrowness' of the gyrochronology relation has hitherto been an unknown; do the three properties, age, mass and rotation period truely lie on an infinitely narrow plane?
Does age depend solely on rotation period and mass, or do other variables influence stellar spin down---perhaps only becoming important after many Myrs?
Metallicity, for example, could play a role---a future study might explore the influence of this property on angular momentum loss rate.
Unfortunately, we cannot answer these questions here: the asteroseismic ages are noisy and observational and intrinsic scatter are ambiguously interwoven.
A future study might include an extra parameter that describes the `width' of the gyrochronological plane and attempt to detect an element of scatter above the noise level.
The success of this approach would strongly depend on having truely representative observational uncertainties.

% THEORY

A rift exists between theory and observation in the field of gyrochronology.
Attempts to unite the two are ongoing \citep{Barnes2011}, \citep{vanSaders2013}.
The gyrochronology relation of \citet{Barnes2010} attemps to reconcile empirical gyrochronology with physical models.
However it requires knowledge of stars' initial rotation periods---in practise, this information is rarely (if ever) available.

% Core-envelope decoupling

{\color{red} to do: read up on sample biases.}
Asteroseismic surveys are biased towards relatively quiet, bright stars, slightly more massive than the Sun.
There are very few young K stars in the \citet{Chaplin2013} sample---presumably because these stars are active.
Unfortunately, those stars that are ideal asteroseismic targets are often less-than-ideal for a rotation survey.

% List of figures:
% marginalised plot
% completeness plot?

\section{Conclusions}
\label{sec:conclusions}

We have calibrated the relation between rotational period, B-V colour and age for MS stars with \teff $<$ 6250 K using 153 Kepler asteroseismic targets, as well as 5 field stars and 260 cluster stars with precise age measurements.
We find that the Sun is no longer well described by gyrochronology---it appears to be slowly rotating for its age and mass compared with the asteroseismic stars.
However it is possible that systematic overestimation of the asteroseismic star's ages, or underestimation of their rotation periods is responsible for this discrepancy (note that the field stars are consistent with the Sun).
This will become clear when more stars in the asteroseismic sample have precise age measurements from individual oscillation mode analysis.
These precise ages will continue to be published over the coming months and years.
Once those stars for which it is possible to measure precise ages (the $\sim$ 150 in the \citet{Chaplin2013} sample with high signal-to-noise), this gyrochronology relation will be updated.

The future looks bright for gyrochronology; K2 will provide rotation periods for hundreds of stars lying in old clusters, with precisely known ages.
These data will be invaluable for gyrochronology and will provide a far more comprehensive view of the period-mass-age plane than has been possible so far.
The TESS spacecraft will also deliver high precision photometry, suitable for both asteroseismic studies and rotation period measurement.
We are entering an era where cluster studies and asteroseismology will deliver precise ages for stars with measureable rotation periods: the picture of gyrochronology will continue to advance over the coming years.

The code used in this project can be found at https://github.com/RuthAngus/Gyro.
We would like to thank Eric Mamajek, Marc Pinsonneault, Sydney Barnes, Steve Kawaler, Jerome Bouvier, Daniel Mortlock and David Hogg for useful insight and discussion.

\bibliographystyle{plainnat}
\bibliography{Gyro_paper}

\begin{figure}[ht]
\begin{center}
\includegraphics[width=6in, clip=true, trim=0 0 0.5in 0]{/Users/angusr/angusr/ACF/ACF2/PDCQ3_output/plots_acf/3223000_full.png}
\caption{\emph{Top:} Quarter 3 PDC-MAP light curve of KIC 3223000. \emph{Middle:} The light curve with linear trends removed and gaps interpolated across. \emph{bottom:} The ACF. This star shows high amplitude variability. We measured a rotation period of 4.86 $\pm$ 0.05 days}
\label{fig:lc}
\end{center}
\end{figure}

\begin{figure}[ht]
\begin{center}
\includegraphics[width=6in, clip=true, trim=0 0 0.5in 0]{/Users/angusr/angusr/ACF/ACF2/PDCQ3_output/plots_acf/3429205_full.png}
\caption{\emph{Top:} Quarter 3 PDC-MAP light curve of KIC 3429205. \emph{Middle:} The light curve with linear trends removed and gaps interpolated across. \emph{bottom:} The ACF. This star does not display high amplitude variability---no rotation period was measured for this target.}
\label{fig:lc}
\end{center}
\end{figure}
%
\begin{figure}[ht]
\begin{center}
\includegraphics[width=6in, clip=true, trim=0 0 0.5in 0]{/Users/angusr/angusr/ACF/ind_qs_figs/3223000.pdf}
\caption{Period measurements for quarters 3 - 16 of star `KIC 7771282'. The blue solid line indicates the median value and the shaded blue region marks its 15\% margin. The blue dashed line with surrounding shaded area indicates one half of the median period measurement with 15\% margin.  More than two thirds of the period measurements were present and consistent to within 15\% of the median for this star, so it passed the selection process.}
\label{fig:ind_qs}
\end{center}
\end{figure}

\end{document}
