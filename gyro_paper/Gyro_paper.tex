\documentclass[12pt,preprint]{aastex}
\usepackage{cite,natbib}
\usepackage{epsfig}
\usepackage{cases}
\usepackage[section]{placeins}
\usepackage{graphicx, subfigure}

\begin{document}

\title{Calibrating Gyrochronology using Kepler Asteroseismic targets}

\author{Ruth Angus$^1$, Suzanne Aigrain$^1$, Amy McQuillan$^2$, William, J. Chaplin$^3$}
\affil{$^1$Department of Physics, University of Oxford, OX1 3RH, UK}
\affil{$^2$School of Physics and Astronomy, Raymond and Beverly Sackler, Faculty of Exact Sciences, Tel Aviv University, 69978, Tel Aviv, Israel}
\affil{$^3$School of Physics and Astronomy, University of Birmingham, Edgbaston, Birmingham, B15 2TT, UK}

\section{Abstract}
\label{abs}

Measuring ages for intermediate and low-mass stars on the main
sequence is challenging, but important for a wide range of studies,
from Galactic dynamics to stellar and planetary evolution. Among the available methods, gyrochronology is a powerful one, because it re- quires knowledge of only the star�s mass (or colour) and its rotation period. However, it is not well calibrated at late ages, and suffers from large uncertainties. Asteroseismology provides relatively precise age measurements for some of the brightest stars observed by Kepler. We measured the photometric rotation periods of 144 stars with asteroseismic ages and compare our results to the existing gyrochronology relations.
We find that the current relations over-predict stellar rotation periods for low-mass stars.   

\section{Introduction}
\label{intro}


Stellar ages are difficult to measure on the main sequence (MS) as their observational properties are not strong functions of age. 
Chromospheric activity is often used as an age indicator, as well as lithium depletion, but the evolution of these properties on the main-sequence is poorly understood, plus, high resolution spectra are required to measure either of these properties. 
Isochrone placement for field stars is a notoriously model-dependent and imprecise dating method, often producing age estimates with uncertainties of order 100 \% or more. 
The gain in precision achieved for fitting isochrones to ensembles of cluster members is substantial; cluster ages can have uncertainties as low as 10\% \citep{Soderblom_2010}. 
However, because the majority of clusters are young, there is a significant lack of precisely measured ages for old stars. 
For this reason the current gyrochronology relations are poorly calibrated at late ages. 
Photometrically determined rotation periods for stars in the older Kepler clusters will provide excellent anchors for gyrochronology, however extracting rotation periods for these stars is difficult due, in part, to Kepler's large pixels. 
The gyrochronology relations of \citet{Mamajek_2008} were calibrated using activity indicators for old field stars as well as vsini measurements of young cluster stars. 
The evolution of stellar activity is currently poorly understood, especially for older stars and whilst the \citet{Mamajek_2008} relation is a revision of the \citet{Barnes_2007} relation, it remains poorly calibrated at late ages. 

\subsection{Angular momentum loss in Main Sequence stars}
Stars lose angular momentum over their MS lifetimes via a magnetised wind that is constrained to rotate with the stellar surface out to the alfven radius (CITATION). 
The strength of the magnetic field at the stellar surface, and therefore the rate of angular momentum loss, depends on the mass and
 rotation period of a star. 
 Due to this dependence all F, G and K stars converge onto
 a unique mass-period-age relation after $\sim$ the age of the Hyades: 650 Myrs \citep{Irwin_2009}.
The gyrochronology paradigm implies that stellar age can be inferred from mass (or colour) and rotation period measurements alone. 
The gyrochronology relation of \citet{Barnes_2007} was determined empirically from observations of the Hyades and other young clusters and tested on binary stars. 
It can be written:

\begin{equation}
\mbox{log}~ t_{gyro} = \frac{1}{n}\left[\mbox{log}~ P - \mbox{log}~ a - b~\mbox{log}(B-V-c)\right] ,
\label{eq:Barnes2007_2}
\end{equation}

where t, B, V and P are time (in Myr), B and V band colours and rotation period 
(in days), respectively. The values of n, a, b and c are tabulated in ...
This relation was further calibrated by Mamajek and Hillenbrand (2008) with chromospheric activity measurements of field stars.
Their relation takes the same form as Barnes (2007) with revised parameters shown in table ....

Gyrochronology relations can only be applied to F, G and K stars. M dwarfs are fully convective; their magnetic fields are not produced by the same dynamo as in more massive stars and they follow a different (poorly understood) spin down process Mcquillan et al (2013). M dwarf ages cannot be calculated via asteroseismology as they do not show high signal-to-noise oscillations. Stars with masses $> 1.3 M_{\odot}$ have shallow convective zones - they are almost fully radiative - and, again, they have a different dynamo-driven magnetic field. These massive stars remain rapidly rotating throughout their brief MS lifetimes and are therefore not suitable gyrochronology targets \citep{Kraft1967}. 

\subsection{Asteroseismic ages}
Asteroseismically determined ages can be very precise, with uncertainties as low as 10\%, and, unlike activity diagnostics, and rotation period measurements the precision of asteroseismic ages are not, themselves age dependent. 
Pressure waves propagating through stars produce periodic brightening and dimming of the stellar surface on timescales of $\approx$ 5 minutes for solar-like stars.
These oscillations can be detected in short-cadence Kepler data, and a fourier transform of the time-series shows a series of peaks, corresponding to each oscillation mode. 
This series of peaks is modulated by a gaussian envelope, the maximum of which is another fundamental asteroseismic observable, $\nu_{max}$
The frequency of each oscillation mode depends on the integrated sound speed along the path through the star and measuring the frequency separation between oscillation modes yields an estimate of the stellar density. 
When combined with spectroscopic observations and compared with theoretical stellar evolution models, measurements of the oscillation mode frequencies can yield stellar ages.  
The asteroseismic properties of 505 stars published in Chaplin et al (2013) were calculated from measurements of the mean large frequency separation, $\Delta\nu$ and the maximum of the gaussian envelope, $\nu_{max}$
These two fundamental asteroseismic parameters can be used, via the scaling relations below, to derive stellar ages, masses and radii.


\begin{equation}
\frac{\Delta\nu}{\Delta\nu_{\odot}} \approx \sqrt{\frac{M/M_{\odot}}{R/R_{\odot}^3}}
\label{eq:delta_nu}
\end{equation}

\begin{equation}
\frac{\nu_{max}}{\nu_{max,\odot}} \approx \frac{M/M_{\odot}}{(R/R_{\odot})^2\sqrt{(T_{eff}/T_{eff,\odot})}}
\label{eq:delta_nu}
\end{equation}

Ages provided in Chaplin et al (2013), calculated from the mean large frequency separation and this scaling relation have uncertainties of  $\approx$ 35\%.
If the frequency of each oscillation mode is measured individually, one can build a density profile of the star and provide a tighter constraint on the evolutionary stage of the star. 
Ages derived from individual oscillation mode measurements can have uncertainties as small as 10\% (CITATION), however this is a still a manual process and therefore takes time.
Although asteroseismology is a potentially precise dating method, it can only be applied to bright stars observed by missions like Kepler that show solar-like oscillations \citep{Chaplin_2011}. 
It is therefore essential that we have a well calibrated dating method like gyrochronology which can be applied to any F, G or K star with a measurable rotation period. 

\section{Rotation Period Measurement}
\label{rotation_period_measurement}

The Kepler light curves of these 505 asteroseismic targets display quasi-periodic variations on timescales corresponding to rotational periods of the stars due to transiting star spots. 
The auto-correlation function (ACF) method developed by McQuillan et al (2012)  was used to measure rotation
periods from Kepler light curves. 
As an alternative to the standard Fourier decomposition and least-squares fitting of
sinusoidal models \citep{Zechmeister}) autocorrelation is much better suited to signals that are not sinusoidal or strictly periodic. It is more effective at
 distinguishing a true signal from its harmonics.  
 For a detailed description of the advantages of the ACF method, see \citet{McQuillan}.
 
An autocorrelation function describes the self-similarity of a light curve at a range of lag times and the highest peak in the ACF (usually also the first peak) is centered on the rotation period of the star. 
In our implementation the first two peaks in the ACF were identified and the central
value of the first peak was temporarily accepted as the period (unless
the second peak was higher than the first in which case {\it its}
central value was taken). 
Note that an important advantage of the ACF method over a periodogram approach is in the differentiability of harmonic signals produced by multiple active regions on the stellar surface from the true periodic signal: these scenarios usually produce ACFs in which the second peak is higher than the first. 
Subsequent peaks in the ACF lying within 10\% of
integer multiples of the period were identified. 
The final period measurement is calculated from the mean separation between peaks lying at integer multiples 
and the error calculated from the distribution of central peak
values. 
In cases where only one peak was present in the ACF, the
central peak value was kept as the period and the error measured from
the width of the peak.  Example ACFs are shown in figure \ref{fig:subfigures} at the end of this document.


We measured the rotation periods of the stars in our
asteroseismic target sample from Kepler quarters 3-16. 
While some
light curves displayed high amplitude, regular flux variations
produced by star spots, others were dominated by random noise or instrumental
systematics. 
To ensure that the periodicities measured were truly
representative of stellar rotation periods, we split the available
light curves into sections, or 'subsets' and computed an ACF for
each subset. 
These subsets were: quarters 3-6, quarters 7-10, quarters
11-14 and individual quarters 3 -16.  

The ACFs of light curves that did not display high amplitude, regular
variation were often populated by many small, unevenly spaced peaks. We
required the height of the selected peak to be greater than zero and significant with respect to the
immediately surrounding region of the ACF, i.e. for the relative
height of the peak to be greater than some value (in this case,
0.1). We also required that more than one peak lying at an integer multiple of the first be present in the ACF. 
In cases where one or more of these criteria were not met, no
period was measured for that particular section of that star's light curve. 

In order for the period measurement of a star to be deemed `reliable',
we stipulated that a period had to be successfully measured in at
least two thirds of the
data subsets. 
We also required that the successful period
measurements lie within 15\% of the median period value, or a harmonic of that
median. 
Of the 505 targets in the original sample, rotation periods of 144 were reliably measured using the above process. 
An example light curve and ACF for KIC 322300 is shown in figure \ref{fig:lc} and quarter-by-quarter period measurements in figure \ref{fig:ind_qs}. 
All but one of the quarters for this star produced an ACF with a significant initial peak and repeating subsequent peaks and of those, all period measurements lie within 15\% of the median value.  



Long cadence PDC-MAP data were used throughout this analysis (\citet{Smith_2012}, \citet{Stumpe_2012}). 
The PDC-MAP data are the product of an initial systematics removal process applied by the Kepler team,
 in which large-scale linear trends are removed in order to improve planet transit search and
modelling capability. 
PDC-MAP data are not, however, optimised to preserve stellar
variability: signal is removed on timescales longer than ~ 30 days (CITATION).
Additionally, light curves still contain large systematic features,
such as the exponential decays that appear after telescope shutdowns, which affect period measurement precision. 


\section{Gyrochronology Calibration}

505 stars with asteroseismically determined values of mass and age were published in \citet{Chaplin2013} 
and rotation periods were successfully measured for 144 of them. 
We also obtained more precise values of mass and age, (from individual oscillation mode analysis) for 7 of the stars in this sample from (CITE VICTOR).
The 144 stars with rotation periods were divided into MS and sub-giant stars based on their logg values presented in Chaplin et al (2013). 
Stars with logg $>$ 4.0 were classified as MS stars and those with logg $<$ 4.0 as sub-giants. 
Of the 144, 57 were categorised as sub-giants and 87 as MS stars. 
The MS stars were further divided according to their mass, stars with masses above the Kraft break, $> 1.3 M_{\odot}$, were discarded as, these stars do not spin down appreciably on the MS.
Figure \ref{fig:results} shows rotation period vs asterosiesmic age for each of the 59 MS stars with masses between 0.8 and 1.3 $M_{\odot}$. 
Older stars do appear to be more rapidly rotating, as predicted. 
The isomass lines of the \citet{Barnes_2007} and \citet{Mamajek_2008} gyrochronology relations appear to roughly describe the slope of the observed period-age relation, but not the y-axis offset, particularly at lower masses. 
The observed dependence of rotation period on mass is not consistent with predictions from the existing gyrochronology relations for low-mass stars.  
The relations over-predict the rotation periods of low-mass stars for a given stellar age.
The triangular outlier at the bottom right of figure \ref{fig:results} is a rapidly rotating, old, low mass, low metallicity object which we believe to be a synchronised binary. 
It is unfortunate that the single high precision target that survived through the rotation period measurement, sub-giant vetting and mass categorisation processes should be a false positive.



A trend between mass and age is visible in figure \ref{fig:results}, green points lie to the left and pink points to the right. 
In figure \ref{fig:m_vs_a} this trend is clear. 
One would expect not to find old, massive stars due to MS turn-off. 
The lack of young, low mass stars is not surprising either as asteroseismic oscillation signals can be overwhelmed by activity. 

Rotation period vs age for the sub-giants is shown in figure \ref{fig:subgiants}. Models of sub-giant rotation evolution from \citet{van_Saders_2013} are shown for comparison. 
Qualitatively speaking the observations show good agreement with the models.  
The relationship between age and rotation shown in these models is, unlike the empirical gyrochronology relations of MS stars, physically motivated. 
Rotational breaking of sub-giants is a simple consequence of angular momentum conservation during the expansion phase and therefore easy to model.


\section{A new gyrochronology relation}

Following the form of Barnes (2007), we fit two separable power laws to the data: one for the age - period and one for the period - mass dependence. 
A fit to the data was obtained using the method described in \citet{Hogg_2010}.

\section{Discussion}

Figure \ref{fig:p_vs_m} shows rotation period as a function of asteroseismic mass for the MS stars. 
In practise stellar ages are estimated by fitting gyrochronological isochrones (gyrochrones) to their rotation periods and masses (or colours).
Again, it is clear that the \citet{Barnes_2007} and \citet{Mamajek_2008} isochrones are over-predicting rotation periods. 
Rotation period is not strongly dependent on age for the more massive stars which is why the two different age populations, shown in pink and blue in figure \ref{fig:p_vs_m} entirely overlap.
The age of star with mass between 1.1 and 1.3 $M_{\odot}$ and rotation period between 0 and 15 days estimated from gyrochrone fitting would suffer from degeneracy: the isochrones are close together in this region of parameter space. 
More reliable ages are obtainable for low-mass, slowly rotating stars, where there is greater separation between gyrochrones, however these stars are under-represented by our sample. 
The larger number of rapid vs slow rotators is a consequence of rotation period measurement bias: with more revolutions per quarter it is much easier to measure short rotation periods.
The larger number of high mass stars is an asteroseismic bias.   

We have seen that the existing gyrochronology relations over-predict the rotation periods of low-mass stars. 
This can be understood as a consequence of core-envelope decoupling. 

Contamination.
Follow up.

Would like to get more precise ages. 

Would like to get more slowly rotating, lower mass stars - K2 and TESS.

\section{Conclusion}


\section{Still to do and Future}

separate into age bins and produce plots. Before and after fitting.

fitting - new gyrochronology relation

Make the sub-giant plot better

swap back to Teff

make isochrones nicer.

Ask Amy to send latest version of her code (and automated version?)

look up what constitutes an evolved star.

Think about whether you want to put any limitations on the ACF itself, or whether the 

Is it more easy to measure the rotation periods for certain stars as a consequence of diff rot?

Would you expect high or low mass stars to have more differential rotation?

\bibliographystyle{plainnat}
\bibliography{Gyro_paper}


\begin{figure}[ht]
\begin{center}
\includegraphics[width=6in, clip=true, trim=0 0 0.5in 0]{/Users/angusr/angusr/ACF/ACF2/PDCQ3_output/plots_acf/3223000_full.png}
\caption{PDC-MAP and 'processed' light curves for KIC 3223000 with the ACF.}
\label{fig:lc}
\end{center}
\end{figure}


\begin{figure}[ht]
\begin{center}
\includegraphics[width=6in, clip=true, trim=0 0 0.5in 0]{/Users/angusr/angusr/ACF/ind_qs_figs/3223000.pdf}
\caption{Period measurements for quarters 3 - 16 of star `KIC 7771282'. The blue solid line indicates the median value and the shaded blue region marks its 15\% margin. The blue dashed line with surrounding shaded area indicates one half of the median period measurement with 15\% margin.  More than two thirds of the period measurements were present and consistent to within 15\% of the median for this star, so it passed the selection process.}
\label{fig:ind_qs}
\end{center}
\end{figure}

\begin{figure}[ht]
\begin{center}
\includegraphics[width=6in, clip=true, trim=0 0 0.5in 0]{/Users/angusr/Python/Gyro/p_vs_a6.png}
\caption{Rotation period vs asteroseismic age for 59 MS stars. The colour scale shows mass. Isomass lines of the \citet{Barnes_2007} and \citet{Mamajek_2008}  gyrochronology relations are shown for different masses as solid and dashed lines respectively. The triangular point is from the Silva Aguirre (CITE) sample.}
\label{fig:results}
\end{center}
\end{figure}

\begin{figure}[ht]
\begin{center}
\includegraphics[width=6in, clip=true, trim=0 0 0.5in 0]{/Users/angusr/Python/Gyro/m_vs_a.png}
\caption{Asteroseismic mass vs asteroseismic age of the MS stars.}
\label{fig:m_vs_a}
\end{center}
\end{figure}

\begin{figure}[ht]
\begin{center}
\includegraphics[width=6in, clip=true, trim=0 0 0.5in 0]{/Users/angusr/Python/Gyro/a_vs_p6.png}
\caption{Rotation period vs asteroseismic age for 57 sub-giants. Models are from \citet{van_Saders_2013}.}
\label{fig:subgiants}
\end{center}
\end{figure}

\begin{figure}[ht]
\begin{center}
\includegraphics[width=6in, clip=true, trim=0 0 0.5in 0]{/Users/angusr/Python/Gyro/p_vs_m2.png}
\caption{Rotation period vs asteroseismic mass.}
\label{fig:p_vs_m}
\end{center}
\end{figure}

\begin{figure}[ht]
\begin{center}
\includegraphics[width=6in, clip=true, trim=0 0 0.5in 0]{/Users/angusr/Python/Gyro/p_vs_m.png}
\caption{Rotation period vs asteroseismic mass.}
\label{fig:p_vs_m}
\end{center}
\end{figure}


\end{document}

