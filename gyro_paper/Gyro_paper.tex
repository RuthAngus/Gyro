\documentclass[12pt,preprint]{aastex}
\usepackage{cite,natbib}
\usepackage{epsfig}
\usepackage{cases}
\usepackage[section]{placeins}
%\usepackage[demo]{graphicx}
%\usepackage{caption}
%\usepackage{subcaption}
\usepackage{graphicx, subfigure}

\begin{document}

\title{Calibrating Gyrochronology using Kepler Asteroseismic targets}

\author{Ruth Angus$^1$, Suzanne Aigrain$^1$, Amy McQuillan$^2$, William, J. Chaplin$^3$}
\affil{$^1$Department of Physics, University of Oxford, OX1 3RH, UK}
\affil{$2$University of Tel Aviv, Israel}
\affil{$3$University of Birmingham, Birmingham, UK}

\section{Abstract}
\label{abs}

Measuring ages for intermediate and low-mass stars on the main
sequence is challenging, but important for a wide range of studies,
from Galactic dynamics to stellar and planetary evolution. Among the available methods, gyrochronology is a powerful one, because it re- quires knowledge of only the star�s mass (or colour) and its rotation period. However, it is not well calibrated at late ages, and suffers from large uncertainties. Asteroseismology provides relatively precise age measurements for some of the brightest stars observed by Kepler. We measured the photometric rotation periods of 144 stars with asteroseismic ages and compare our results to the existing gyrochronology relations.
We find that the current relations over-predict stellar rotation periods for low-mass stars.   

\section{Introduction}
\label{intro}


The ages of main-sequence (MS) stars are difficult to measure as the majority of their observational properties are not a strong function of age. 
hromospheric activity is often used as an age indicator, as well as lithium depletion, but the evolution of these properties on the main-sequence is poorly understood, plus, high resolution spectra are required to measure either of these properties. 
Isochrone placement for field stars is a notoriously model-dependent and imprecise dating method, often producing age estimates with uncertainties of order 100 \% or more. 
The gain in precision achieved for fitting isochrones to cluster members is substantial, cluster ages can have uncertainties as low as 10\% (CITATION). 
However, the majority of nearby clusters are young, and measuring the rotational velocity of cluster members is extremely difficult for distant clusters. 
It is for this reason that the existing gyrochronology relations remain poorly calibrated for old stars - they rely chiefly on ages of nearby, young clusters. 
Photometrically determined rotation periods for stars in the older Kepler clusters will provide excellent anchors for gyrochronology, however extracting rotation periods for these stars is difficult due, in part, to Kepler's large pixels. 
The gyrochronology relations of Mamajek and Hillenbrand have been calibrated using activity measurements of older field stars as well as cluster stars. 
The evolution of stellar activity is currently poorly understood, especially for older stars and whilst the Mamajek and Hillenbrand relation is a revision of the Barnes (2007) relation, it remains poorly calibrated at late ages. 
Chromospheric activity and rotation period are undoubtably linked (CITATION) and with the photometric precision provided by Kepler, rotation periods are easily measurable for many stars without the need for spectroscopic data. 

\subsection{Angular momentum loss in Main Sequence stars}
Stars lose angular momentum over their MS lifetimes via a magnetised wind that is constrained to rotate with the stellar surface out to the alfven radius (CITATION). 
The strength of the magnetic field at the stellar surface, and therefore the rate of angular momentum loss, depends on the mass and
 rotation period of a star. 
 Due to this dependence rapid rotators spin down faster than slow rotators and all F, G and K stars converge onto
 a unique mass-period-age relation after $\sim$ the age of the Hyades: 650 Myrs \citep{Irwin_2009}.
The gyrochronology paradigm implies that stellar age can be inferred from mass (or colour) and rotation period measurements alone. 
The gyrochronology relation of \citet{Barnes_2007} was determined empirically from observations of the Hyades and other young clusters and tested on binary stars. 
It can be written:

\begin{equation}
\mbox{log}~ t_{gyro} = \frac{1}{n}\left[\mbox{log}~ P - \mbox{log}~ a - b~\mbox{log}(B-V-c)\right] ,
\label{eq:Barnes2007_2}
\end{equation}

where t, B, V and P are time (in Myr), B and V band colours and rotation period 
(in days), respectively. The values of n, a, b and c are tabulated in ...
This relation was further calibrated by Mamajek and Hillenbrand (2008) with chromospheric activity measurements of field stars.
Their relation takes the same form as Barnes (2007) with revised parameters shown in table ....

\subsection{Asteroseismic ages}
Asteroseismically determined ages can be very precise, with uncertainties as low as 10\%, and, unlike activity diagnostics, and rotation period measurements the precision of asteroseismic ages are not, themselves age dependent. 
Pressure waves propagating through stars produce periodic brightening and dimming of the stellar surface on timescales of $\approx$ 5 minutes for solar-like stars.
These oscillations can be detected in short-cadence Kepler data, and a fourier transform of the time-series shows a series of peaks, corresponding to each oscillation mode. 
This series of peaks is modulated by a gaussian envelope, the maximum of which is another fundamental asteroseismic observable, $\nu_{max}$
The frequency of each oscillation mode depends on the integrated sound speed along the path through the star and measuring the frequency separation between oscillation modes yields an estimate of the stellar density. 
When combined with spectroscopic observations and compared with theoretical stellar evolution models, measurements of the oscillation mode frequencies can yield stellar ages.  
The asteroseismic properties of 505 stars published in Chaplin et al (2013) were calculated from measurements of the mean large frequency separation, $\Delta\nu$ and the maximum of the gaussian envelope, $\nu_{max}$
These two fundamental asteroseismic parameters can be used, via the scaling relations below, to derive stellar ages, masses and radii.


\begin{equation}
\frac{\Delta\nu}{\Delta\nu_{\odot}} \approx \sqrt{\frac{M/M_{\odot}}{R/R_{\odot}^3}}
\label{eq:delta_nu}
\end{equation}

\begin{equation}
\frac{\Delta\nu}{\Delta\nu_{\odot}} \approx \sqrt{\frac{M/M_{\odot}}{R/R_{\odot}^3}}
\label{eq:delta_nu}
\end{equation}

Ages provided in Chaplin et al (2013), calculated from the mean large frequency separation and this scaling relation have uncertainties of  $\approx$ 35\%.
If the frequency of each oscillation mode is measured individually, one can build a density profile of the star and provide a tighter constraint on the evolutionary stage of the star. 
Ages derived from individual oscillation mode measurements can have uncertainties as small as 10\% (CITATION), however this is a still a manual process and therefore takes time.
Although asteroseismology is a potentially precise dating method, it can only be applied to bright stars observed by missions like Kepler that show solar-like oscillations \citep{Chaplin_2011}. 
It is therefore essential that we have a well calibrated dating method like gyrochronology which can be applied to any F, G or K star with a measurable rotation period. 

\section{Rotation Period Measurement}
\label{rotation_period_measurement}

The Kepler light curves of these 505 asteroseismic targets display quasi-periodic variations on timescales corresponding to rotational periods of the stars due to transiting star spots. 
The auto-correlation function (ACF) method developed by McQuillan et al (2012)  was used to measure rotation
periods from Kepler light curves. 
As an alternative to the standard Fourier decomposition and least-squares fitting of
sinusoidal models \citep{Zechmeister}) autocorrelation is much better suited to signals that are not sinusoidal or strictly periodic. It is more effective at
 distinguishing a true signal from its harmonics.  
 For a detailed description of the advantages of the ACF method, see \citet{McQuillan}.
 
An autocorrelation function describes the self-similarity of a light curve at a range of lag times and the highest peak in the ACF (usually also the first peak) is centered on the rotation period of the star. 
In our implementation the first two peaks in the ACF were identified and the central
value of the first peak was temporarily accepted as the period (unless
the second peak was higher than the first in which case {\it its}
central value was taken). 
Note that an important advantage of the ACF method over a periodogram approach is in the differentiability of harmonic signals produced by multiple active regions on the stellar surface from the true periodic signal: these scenarios usually produce ACFs in which the second peak is higher than the first. 
Subsequent peaks in the ACF lying within 10\% of
integer multiples of the period were identified. 
The final period measurement is calculated from the mean separation between peaks lying at integer multiples 
and the error calculated from the distribution of central peak
values. 
In cases where only one peak was present in the ACF, the
central peak value was kept as the period and the error measured from
the width of the peak.  Example ACFs are shown in figure \ref{fig:subfigures} at the end of this document.


We measured the rotation periods of the stars in our
asteroseismic target sample from Kepler quarters 3-16. 
While some
light curves displayed high amplitude, regular flux variations
produced by star spots, others were dominated by random noise or instrumental
systematics. 
To ensure that the periodicities measured were truly
representative of stellar rotation periods, we split the available
light curves into sections, or 'subsets' and computed an ACF for
each subset. 
These subsets were: quarters 3-6, quarters 7-10, quarters
11-14 and individual quarters 3 -16.  

The ACFs of light curves that did not display high amplitude, regular
variation were often populated by many small, unevenly spaced peaks. We
required the height of the selected peak to be greater than zero and significant with respect to the
immediately surrounding region of the ACF, i.e. for the relative
height of the peak to be greater than some value (in this case,
0.1). We also required that more than one peak lying at an integer multiple of the first be present in the ACF. 
In cases where one or more of these criteria were not met, no
period was measured for that particular section of that star's light curve. 

In order for the period measurement of a star to be deemed `reliable',
we stipulated that a period had to be successfully measured in at
least two thirds of the
data subsets. 
We also required that the successful period
measurements lie within 15\% of the median period value, or a harmonic of that
median. 
Of the 165 targets in the original sample, rotation periods of 43 were reliably measured using the above process. 
Results are shown for an example star (KIC 7771282) in figure \ref{fig:ind_qs}. 
All but one of the quarters for this star produced an ACF with a significant initial peak and repeating subsequent peaks and of those, all period measurements lie within 15\% of the median value.  

\begin{figure}[ht]
\begin{center}
\includegraphics[width=6in, clip=true, trim=0 0 0.5in 0]{/Users/angusr/angusr/ACF/ind_qs_figs/7771282.pdf}
\caption{Period measurements for quarters 3 - 16 of star `KIC 7771282'. The blue solid line indicates the median value and the shaded blue region marks its 15\% margin. The blue dashed line with surrounding shaded area indicates one half of the median period measurement with 15\% margin. The vertical red dashed line indicates that no period was measured for that quarter. More than two thirds of the period measurements were present and consistent to within 15\% of the median for this star, so it passed the selection process.}
\label{fig:ind_qs}
\end{center}
\end{figure}


Long cadence PDC-MAP data were used throughout this analysis (\citet{Smith_2012}, \citet{Stumpe_2012}). 
The PDC-MAP data are the product of an initial systematics removal process applied by the Kepler team,
 in which large-scale linear trends are removed in order to improve planet transit search and
modelling capability. 
PDC-MAP data are not, however, optimised to preserve stellar
variability: signal is removed on timescales longer than ~ 30 days (CITATION).
Additionally, light curves still contain large systematic features,
such as the exponential decays that appear after telescope shutdowns, which affect our
ability to measure accurate rotation periods. 

\section{Method Validation}
\label{method_validation}

In order to test the completeness and reliability of our method we applied the autocorrelation and selection process outlined above to 1000 simulated light curves. 
Synthetic stellar light curves were generated with varying rotation periods and spot lifetimes. 
Other parameters, such as number of spots and inclination were held fixed. Differential rotation was not included in the model. 
10 `quiet' stars were selected from the original sample (based on a visual examination of their light curves). For each of these real light curves, 100 simulated light curves with random periods (drawn from a log-space distribution, uniform between 1 and 100) and random spot lifetimes (drawn from a log-space distribution, uniform between 1 and 10) were injected with random scaling factors (uniform in log-space between 1 and 15) - generating a total of 1000 light curves with random (but documented) periods, spot lifetimes and amplitudes.
ACFs were calculated for these light curves and the measured periods compared with their true values. 
The results are shown in figure \ref{fig:compare}. It is clear from this figure that we are systematically underestimating rotation periods and that the offset is worse for longer periods. 
This effect has beed noted previously and may be due to .... 
No rotation periods greater than 23 days were measured --- this is limited by the 90 day long Kepler quarters (in fact it doesn't make sense to generate false light curves with periods longer than this!). 
The origin of the small group of points clustered in the bottom right remains a mystery and needs further investigation.


\begin{figure}[ht]
\begin{center}
\includegraphics[width=6in, clip=true, trim=0 0 0.5in 0]{/Users/angusr/Python/K-ACF/method_val/Compare.png}
\caption{A comparison of measured and true rotation periods for simulated light curves. Dotted lines show multiples and factors of 2. Rotation periods larger than 21 days were not measurable with the single quarter method. }
\label{fig:compare}
\end{center}
\end{figure}

Figures \ref{fig:completeness} shows the completeness, reliability and contamination of the sample as a function of period, spot lifetime and amplitude. 
Completeness, reliability and contamination are also shown for each star in figure \ref{fig:completeness}. 
Completeness is defined as  the percentage of the sample with successful period measurement, reliability is defined as the fraction of sample for which the period measurement lies within 20\% of the true value and contamination is the fraction of objects for which the period measurement differed from the true value by greater than 20\%.

\begin{figure}[ht]
\begin{center}
\includegraphics[width=6in, clip=true, trim=0 0 0.5in 0]{/Users/angusr/Python/K-ACF/method_val/Completeness.png}
\caption{Completeness, reliability and contamination vs rotation period for the 1000 simulated light curves. }
\label{fig:completeness}
\end{center}
\end{figure}

\section{Gyrochronology Calibration}

Rotational periods were successfully measured for 43 of the original
165 stars. Figure \ref{fig:results} shows the rotation periods measured for these 43
stars with their B - V colours, taken from \citet{catalogue}. The
coloured lines represent isochrones of varying ages and were
calculated using equation \ref{eq:Barnes2007_2}.

\begin{figure}[ht]
\begin{center}
\includegraphics[width=6in, clip=true, trim=0 0 0.5in 0]{Gyro_results.png}
\caption{Rotation period measurements for 43 stars shown with their
  Kepler input catalogue colours. Isochrones were calculated from
  equation \ref{eq:Barnes2007_2} \citep{Barnes_2007}.}
\label{fig:results}
\end{center}
\end{figure}

505 stars with asteroseismically determined values of mass and age were published in Chaplin (2013) (CITE THIS PROPERLY). 
We successfully measured the rotation periods of 145 of these stars. 
A plot of rotation period vs asteroseismic age is shown in figure \ref{fig:p_vs_a}.


\begin{figure}[ht]
\begin{center}
\includegraphics[width=6in, clip=true, trim=0 0 0.5in 0]{/Users/angusr/Python/Gyro/p_vs_a.png}
\caption{Photometric rotation period vs asteroseismic age for the 145 stars with successful rotation period measurements. The three colours show three different mass bins. Isomass lines are calculated from the Barnes (2007) gyrochronology relations. There is no isomass line for the most massive bin as the Barnes relation does not apply to stars of this mass. }
\label{fig:p_vs_a}
\end{center}
\end{figure}

We obtained more precise values of mass and age for 7 stars from (CITE VICTOR).
These new data points are shown in figure \ref{p_vs_a2}. 

\begin{figure}[ht]
\begin{center}
\includegraphics[width=6in, clip=true, trim=0 0 0.5in 0]{/Users/angusr/Python/Gyro/p_vs_a2.png}
\caption{Photometric rotation period vs asteroseismic age for the 145 stars with successful rotation period measurements. The three colours show three different mass bins. Isomass lines are calculated from the Barnes (2007) gyrochronology relations. There is no isomass line for the most massive bin as the Barnes relation does not apply to stars of this mass. }
\label{fig:p_vs_a2}
\end{center}
\end{figure}

\section{Still to do and Future}

separate into age bins 

fitting

replace figures

Either cut out injection tests or make them better.

3. More injections? / injection tests for year by year / redo injections, this time making sure that the light curves are continuous

4. Ask Amy to send her latest version of her code (and automated version?)

5. Figure out what's going wrong with injections

6. look up what constitutes an evolved star.

8. PCA?

9. HARPS North targeted asteroseismology RV planet search proposal.

10. Build year by year data into injection tests. 


model star spot light curves using a quasi-periodic GP. 
marginalise over all parameters except period. 
If you had a good enough model you could model spot lifetimes, differential rotation, inclination, filling factor and rotation period and just marginalise over everything but period to get the maximum marginalised likelihood. 
Since the physical model is not well understood you could effectively do the same thing but with a non-parametric model. 

Find the underlying trend via PCA


\bibliographystyle{plainnat}
\bibliography{Gyro_paper}

\end{document}

