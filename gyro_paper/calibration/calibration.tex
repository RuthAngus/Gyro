\documentclass[10pt,preprint]{aastex}

\usepackage{amsmath}
\usepackage{breqn}
\usepackage{cite,natbib}
\usepackage{natbib}
\usepackage{epsfig}
\usepackage{cases}
\usepackage[section]{placeins}
\usepackage{graphicx, subfigure}
\usepackage{color}
\usepackage{amsmath}
\usepackage{float}
\floatplacement{figure}{H}
% \usepackage[nomarkers,figuresonly]{endfloat}

\newcommand{\logg}{log \emph{g}~}
\newcommand{\teff}{$T_{eff}~$}
\newcommand{\prot}{$P_{rot}~$}

\newcommand{\ah}{$\hat{A}_n$}
\newcommand{\ph}{$\hat{P}_n$}
\newcommand{\ch}{$\hat{C}_n$}
\newcommand{\gh}{$\hat{G}_n$}
\newcommand{\yh}{$\hat{Y}_n$}
\newcommand{\teffh}{$\hat{T}_n$}

\newcommand{\feh}{[Fe/H]}
\newcommand{\dd}{\ensuremath{\,\mathrm{d}}}

\begin{document}

\section{Calibrating the Gyrochronology relation}
\label{sec:gyro_cal}

\subsection{The model}

The 153 asteroseismic stars in our sample have B-V colours converted from effective temperatures, photometric rotation periods, P and asteroseismically derived ages, A and surface gravities, \logg (G).
Each of these properties has an associated uncertainty, assumed to be independent and Gaussian for B-V and P and log-normal for A and G.
The 265 cluster and field stars added to our sample do not have \logg values; however, since we only use \logg to separate the populations of subgiants and dwarfs (and we assume that the cluster and field stars are dwarfs) this shouldn't hurt our analysis.

Hot stars and subgiants follow a different gyrochronology to MS dwarfs.
Stars with effective temperatures above the Kraft-break, $T_{eff} \sim$ 6250 K, \citep{Kraft1967} do not have a thick convective envelope and cannot support a strong magnetic dynamo, so do not spin down appreciably during their MS lifetimes.
Subgiants spin down rapidly as they expand, due to angular momentum conservation, and thus diverge from the gyrochronological mass-period-age plane.
The point in their evolution at which they turn off the `gyrochronological MS turnoff' is difficult to define.
Classically, MS turnoff is defined as the hottest point on a star's path on the HR diagram (before it ascends the giant branch) but theory predicts that evolving stars begin the process of spinning down relatively slowly after leaving the `classically defined' MS \citep{vanSaders2013}.
For this reason we choose a very simple definition of MS turnoff---we use a \logg cut of 4.0 to differentiate dwarfs from giants.

We do not simply exclude hot stars and subgiants from our sample during the modelling process---we model all three populations simultaneously.
This allows for the fact that stars have some probability mass lying in all three regimes due to their large observational uncertainties.

In addition to hot stars and subgiants, there is further population of contaminating stars in our sample: rapid rotators.
Around 20 stars with rotation periods of less than $\sim$ 5 days do not lie on the standard gyrochronology mass-period-age plane.
These stars could plausibly be synchronised binaries or stars with unseen, close-in, massive planets (\citet{Poppenhaeger2014}, \citet{Beky2014})
{\color{red} check to see if this population still exists and potentially take this out}

We model four stellar populations simultaneously: cool MS stars, hot MS stars, subgiants and rapid rotators.
The hot MS stars are defined as those with B-V $<$ 0.45, corresponding to \teff $\approx$ 6250 K for Solar metallicity and \logg.
Since there is no dependence of age on rotation period for massive MS stars, their ages are modelled as a log-normal distribution with mean and standard deviation, Y and V, free parameters.
Subgiant ages \emph{do} depend on period and $T_{eff}$; however, since we are not interested in the rotational properties of these stars for the purposes of gyrochronology calibration, we also model them with a log-normal distribution with mean and standard deviation, Z and U, free parameters.
% We could model the subgiants with an analytic expression for age, given colour and period, such as the one in \citet{vanSaders2013}, however, we would like to remain as model \emph{independent} as possible throughout this process.

We use a mixture model for the remaining two populations of cool MS stars: those that follow the gyrochronology relation and those with unusually fast rotation periods.
The fast rotators are treated as 'outliers' and modelled with another log normal distribution with mean and standard deviation, X and U, again inferred from the data.
In our mixture model an additional parameter, $P_b$, is the probability of each data point being an outlier.

Ideally both the hot star (B-V $<$ 0.45) and subgiant (\logg $<$ 4) boundaries would be free parameters in our model.
However, since these two populations are modelled with a relatively unconstraining log-normal distribution, these boundary parameters would not be well behaved.
Both would be pushed to higher and higher values until all stars were modelled with a log normal distribution.
In order to avoid this problem, we fixed these two boundaries.
A future analysis could deal with this issue by maximising likelihood over a grid of boundary parameter values.
% {\color{red}Try using different values and seeing how sensitive you are!}
Alternatively, one could avoid the assumption that the gyrochronology relation is infinitely narrow and assign it some intrinsic width, which would also be a free parameter.

Our likelihood function for the cool dwarf regime can be written as follows:

% \begin{eqnarray}
% 	\mathcal{L}_{gyro} = \prod_{i=1}^N \left[ \frac{1-P_b}{\sqrt{2\pi\sigma_{i}^2}}
% 	\exp\left({-\frac{\left(\sum_{j=1}^J [A_{i}- (P_{ij} \times \frac{1}{a}([B-V]_{ij}-c)^{-b})^{\frac{1}{n}}]\right)^2} {2\sigma_{i}^2}} \right)  \right] \nonumber \\
% 	+ \frac{P_b}{\sqrt{2\pi[U+\sigma_{i}^2]}} \exp \left( -\frac{[A_i - X]^2}{2[U+\sigma_{i}^2]}  \right)
% \end{eqnarray}
% \label{eq:likelihood}

\begin{eqnarray}
	\mathcal{L}_{gyro} = \prod_{i=1}^N \left[ \frac{1-P_b}{\sqrt{2\pi\sigma_{i}^2}}
	\exp\left({-\frac{\left[P_i - A_i^n \times a(B_i-V_i-c)^b\right]^2}{2\sigma_{i}^2}} \right)  \right] \nonumber \\
	+ \frac{P_b}{\sqrt{2\pi[U+\sigma_{i}^2]}} \exp \left( -\frac{[A_i - X]^2}{2[U+\sigma_{i}^2]}  \right)
\end{eqnarray}
\label{eq:likelihood}

where $P_b$ is the probability that a star is drawn from the background log-normal distribution, with mean, X and standard deviation, U.

The likelihood function for the hot and subgiant regimes can be written

\begin{eqnarray}
	\mathcal{L}_{hot} = \prod_{i=1}^N \left[ \frac{1}{\sqrt{2\pi[V+\sigma_{i}^2]}}
	\exp\left({-\frac{\left(A_{i}- Y\right)^2} {2[V+\sigma_{i}^2]}} \right)  \right] \\
	\nonumber \\
	\mathcal{L}_{subgiant} = \prod_{i=1}^N \left[ \frac{1}{\sqrt{2\pi[W+\sigma_{i}^2]}}
	\exp\left({-\frac{\left(A_{i}- Z\right)^2} {2[W+\sigma_{i}^2]}} \right)  \right] \nonumber \\
\end{eqnarray}

{\color{red} why should the Ns be different? It's the same N, surely.}

respectively, where Y and Z are the respective means and V and W the respective standard deviations.
The likelihood can be written as the sum of likelihoods over the three different regimes:%, k:

\begin{equation}
	\mathcal{L} = \mathcal{L}_{gyro} + \mathcal{L}_{hot} + \mathcal{L}_{sub}
\end{equation}

\subsection{Accounting for observational uncertainties}

We postulate that there is a deterministic relationship between the "true" rotation period, $P_n$, of a star and its "true" age, $A_n$, and colour, $C_n$, described by equation \ref{eq:plane}.
By "true" we mean the value an observable property would take, given infinitely high signal-to-noise measurements.
$P_n$ also depends on \logg($G_n$) since this property determines whether a star falls in the dwarf or subgiant regime.

We would like to sample the posterior probability of the parameter vector $\theta = \{a, b, n, X, Y, Z, U, V, W, P_b\}$ conditioned on a set of noisy observations \ph, \ah, \ch, and \gh and therefore need to compute the marginalised likelihood,

\begin{equation}
	p(\{\hat{P}_n,\hat{A}_n,\hat{C}_n,\hat{G}\}|\theta) =
	\prod_{n=1}^{n} \int p(\hat{P}_n,\hat{A}_n,\hat{C}_n,\hat{G}_n,P_n,A_n,C_n,G_n|\theta)
	{\rm d}P_n {\rm d}A_n {\rm d}C_n,{\rm d}G_n
\label{eq:fulll}
\end{equation}

The joint probability can be factorised as

\begin{align}
	p(\hat{P}_n,\hat{A}_n,\hat{C}_n,\hat{G}_n,P_n,A_n,C_n,G_n\,|\,\theta) = & \nonumber \\
	p(P_n)\,p(C_n)\,p(G_n)\,p(P_n\,| & \,A_n,C_n,G_n,\theta)\
        p(\hat{P}_n\,|\,P_n)\,p(\hat{A}_n\,|\,A_n)\,p(\hat{C}_n\,|\,C_n)\,p(\hat{G}_n\,|\,G_n)
\nonumber
\end{align}

where, in the cool dwarf regime
\begin{eqnarray}
p(P_n\,|\,A_n,C_n,G_n,\theta) =
	& (1-P_b)~\delta \left [P_n - \left(\left[A^n \times a(B-V - c)^b\right]^n\right) \right] \quad \\
	& +~P_b~\times \left(\sqrt{2\pi[U^2+\sigma^2]}\right)^{-1/2}~\exp\left({\frac{(P_n-X)^2}{2[U^2+\sigma^2]}}\right),
\end{eqnarray}
in the hot dwarf regime
\begin{eqnarray}
p(P_n\,|\,A_n,C_n,G_n,\theta) = \left(\sqrt{2\pi[V^2+\sigma^2]}\right)^{-1/2}~\exp\left({\frac{(P_n-Y)^2}{2[V^2+\sigma^2]}}\right)
\end{eqnarray}
and in the subgiant regime
\begin{eqnarray}
p(P_n\,|\,A_n,C_n,G_n,\theta) = \left(\sqrt{2\pi[W^2+\sigma^2]}\right)^{-1/2}~\exp\left({\frac{(P_n-Z)^2}{2[W^2+\sigma^2]}}\right),
\end{eqnarray}

We can compute equation \ref{eq:fulll}, up to an unimportant constant, using a sampling approximation.
The values of \ah, \ch~ (or \teffh) and \gh with uncertainties, $\sigma_A$, $\sigma_C$ and $\sigma_G$, reported in catalogues provide constraints on the posterior probability of those variables, under a choice of prior.
Ideally, these catalogues would provide samples from their posterior PDFs for the asteroseismically determined parameters \ah, \teffh and \gh which we could use directly.
i.e. samples from
\begin{equation}
p(\hat{A}_n, \hat{C}_n, \hat{G}_n|D_n) = \frac{1}{\beta_n}p(D_n|\hat{A}_n, \hat{C}_n, \hat{G}_n)p_0(\hat{A}_n, \hat{C}_n, \hat{G}_n)
\end{equation}
where $p(D_n|\hat{A}_n, \hat{C}_n, \hat{G}_n)$ is the likelihood of the data, $D_n$ (in this case, the set of Kepler lightcurves and supplementary \teff and \feh values), $\beta_n$ is a normalisation constant and $p_0(\hat{A}_n, \hat{C}_n, \hat{G}_n)$ is an uninformative prior PDF, chosen by the fitter (\citet{Chaplin2013} use a flat prior in age and log $g$).

In the absence of posterior PDF samples we generate our own from Gaussian distributions with means, \ah, \ch, \gh and standard deviations, $\sigma_A^2$, $\sigma_C^2$ and $\sigma_G^2$.
We generate $J_n$ posterior samples:
\begin{eqnarray}
A_n^{(j)} &\sim& p(A_n\,|\,\hat{a}_n) \nonumber \\
C_n^{(j)} &\sim& p(C_n\,|\,\hat{C}_n) \nonumber \\
G_n^{(j)} &\sim& p(G_n\,|\,\hat{G}_n)
\end{eqnarray}
and use these to evaluate $p(\hat{P_n}|P_n)$ up to a normalisation constant.
We then evaluate the marginalized likelihood for a single star as follows

\begin{align}
	p(\hat{P}_n,\hat{A}_n,\hat{C}_n,\hat{G_n}\,|\,\theta) = & \nonumber\\
\int
p(P_n)\,p(C_n)\, & p(G_n)\,p(P_n\,|\,A_n,C_n,G_n,\theta)\,
	p(\hat{P}_n\,|\,P_n)\,p(\hat{A}_n\,|\,A_n)\,p(\hat{C}_n\,|\,C_n),p(\hat{G}_n\,|\,G_n)
    \dd P_n \dd A_n \dd C_n \dd G_n \nonumber\\
&\propto \int
    p(P_n\,|\,A_n,C_n,G_n,\theta)\,p(\hat{P}_n\,|\,P_n)\,
    p(A_n\,|\,\hat{A}_n)\,p(C_n\,|\,\hat{C}_n),p(G_n\,|\,\hat{G}_n)
    \dd P_n \dd A_n \dd C_n \dd G_n \nonumber\\
&\approx \frac{1}{J_n} \sum_{j=1}^{J_n}p(\hat{P}_n\,|\,P_n^{(j)})
\end{align}

where $P_n^{(j)}$ is computed from the posterior samples.
Finally, the full marginalized log-likelihood is
\begin{eqnarray}
	\log p(\{\hat{P}_n,\hat{A}_n,\hat{C}_n,\hat{G}_n\}\,|\,\theta) &\approx&
    \log\zeta + \sum_{n=1}^N
        \log \left[ \sum_{j=1}^{J_n}p(\hat{P}_n\,|\,P_n^{(j)}) \right ]
\end{eqnarray}
where $\zeta$ is an irrelevant normalization constant and $p(\hat{P}_n|P_n)$ is defined in \ref{eq:likelihood}.
We used {\tt emcee} \citep{Foreman-Mackey2013}, an affine invariant, ensemble sampler Markov Chain Monte Carlo (MCMC) algorithm, to explore the posterior probability distributions of the model parameters, $\theta$.

\bibliographystyle{plainnat}
\bibliography{Gyro_paper}

\end{document}
